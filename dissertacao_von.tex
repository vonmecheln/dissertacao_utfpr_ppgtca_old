%\documentclass[openright]{normas-utf-tex} %openright = o capitulo comeca sempre em paginas impares
\documentclass[oneside]{normas-utf-tex} %oneside = para dissertacoes com numero de paginas menor que 100 (apenas frente da folha) 


% force A4 paper format
\special{papersize=210mm,297mm}

\usepackage[
%pdftex,pdfauthor={Your Name},
%pdftitle={The Title},
%pdfsubject={The Subject},
%pdfkeywords={Some Keywords},
%pdfproducer={Latex with hyperref, or other system},
%pdfcreator={pdflatex, or other tool},
hidelinks,backref]{hyperref} % gera hiperlinks para o sumario, links, referencias -- deve vir antes do 'abntcite' 

\usepackage[alf,abnt-emphasize=bf,bibjustif,recuo=0cm, abnt-etal-cite=2, abnt-etal-list=99]{abntcite} %configuracao correta das referencias bibliograficas.

\usepackage[brazil, english, latin]{babel} % pacote portugues brasileiro
\usepackage[latin1]{inputenc} % pacote para acentuacao direta
\usepackage{amsmath,amsfonts,amssymb} % pacote matematico
\usepackage{graphicx} % pacote grafico
\usepackage{times} % fonte times
\usepackage[final]{pdfpages} % adicao da ata
\usepackage{nameref}

%\usepackage[scaled]{helvet}
\renewcommand*\familydefault{\sfdefault} %% Only if the base font of the document is to be sans serif
\usepackage[T1]{fontenc}


\usepackage[disable]{todonotes} % adicionar parâmetro [disable] para ocultar os comentários
\usepackage{xspace} % usado nos comentários
%\usepackage{microtype} % faz refinamento de tipografia 
%\usepackage{multirow} % faz múltiplas linhas na tabela
%\usepackage{listings} % Include the listings-package
%\usepackage{float} % fixa a posição da figura/tabela com H

\usepackage{chngcntr} % Corrige o contador de figuras e tabelas
\counterwithin{figure}{chapter}
\counterwithin{table}{chapter}

\usepackage{tabularx}
\usepackage{longtable}

% atalho para \ref
\newcommand{\fig}[1]{Figura~\ref{#1}}

% toda vez que inserir palavras em ingles, usar esse comando: \ew{palavraEmIngles}
\newcommand{\ew}[1]{\selectlanguage{english}\textit{#1}\selectlanguage{brazil}}

% toda vez que inserir palavras em latim, usar esse comando: \lw{palavraEmLatin}
\newcommand{\lw}[1]{\selectlanguage{latin}\emph{\textit{#1}}\selectlanguage{brazil}}

\newcounter{todocounter}
\setlength{\marginparwidth}{2cm}
\reversemarginpar
\newcommand{\nota}[1]{\xspace\stepcounter{todocounter}\todo[fancyline]{\thetodocounter: #1}}
\newcommand{\notain}[2]{\stepcounter{todocounter}\todo[inline]{\thetodocounter: (#1) #2}}
\newcommand{\notahl}[2]{\xspace\stepcounter{todocounter}\texthl{#1}\xspace\todo[fancyline]{\thetodocounter: #2}}

\newcommand{\notaAlu}[1]{\xspace\stepcounter{todocounter}\todo[fancyline, color=green!50]{{\tiny[Aluno]} \scriptsize#1}}
\newcommand{\notaOri}[1]{\xspace\stepcounter{todocounter}\todo[fancyline, color=blue!30]{{\tiny[Orientador]} \scriptsize#1}}


\usepackage{soul} %texto tachado
 
%Podem utilizar GEOMETRY{...} para realizar pequenos ajustes das margens. Onde, left=esquerda, right=direita, top=superior, bottom=inferior. P.ex.:
%\geometry{left=3.0cm,right=1.5cm,top=4cm,bottom=1cm} 

% ---------- Preambulo ----------
\instituicao{Universidade Tecnol\'ogica Federal do Paran\'a} % nome da instituicao
\programa{Programa de P\'{o}s-Gradua\c{c}\~{a}o em Tecnologias Computacionais para o Agroneg\'{o}cio} % nome do programa
\area{Inform\'atica Industrial} % [Engenharia Biom\'edica] ou [Inform\'atica Industrial] ou [Telem\'atica]

\documento{Disserta\c{c}\~ao} % [Disserta\c{c}\~ao] ou [Tese]
\nivel{Mestrado} % [Mestrado] ou [Doutorado]
\titulacao{Mestre} % [Mestre] ou [Doutor]

\titulo{{Geração de banco de imagens padronizadas de teste de tetrazólio nas sementes de soja para identificação de danos através de técnicas de  processamento digital de imagens}} % titulo do trabalho em portugues
\title{\MakeUppercase{Title in English}} % titulo do trabalho em ingles

\autor{Luís Henrique Manosso Von Mecheln} % autor do trabalho
\cita{MECHELN, Luís Henrique Manosso Von} % sobrenome (maiusculas), nome do autor do trabalho

\palavraschave{controle de qualidade, dados de sementes de soja, teste de tetrazólio, classificação, recursos visuais} % palavras-chave do trabalho
\keywords{quality control, soybean seed data, tetrazolium test, classication , visual features} % palavras-chave do trabalho em ingles

%\comentario{\UTFPRdocumentodata\ apresentada ao \UTFPRprogramadata\ da \ABNTinstituicaodata\ como requisito parcial para obten\c{c}\~ao do grau de ``\UTFPRtitulacaodata\ em Ci\^encias'' -- \'Area de Concentra\c{c}\~ao: \UTFPRareadata.}

\orientador{Prof. Dr. Paulo Lopes de Menezes} % nome do orientador do trabalho
%\orientador[Orientadora:]{Nome da Orientadora} % <- no caso de orientadora, usar esta sintaxe
%\coorientador{Nome do Co-orientador} % nome do co-orientador do trabalho, caso exista
%\coorientador[Co-orientadora:]{Nome da Co-orientadora} % <- no caso de co-orientadora, usar esta sintaxe
%\coorientador[Co-orientadores:]{Nome do Co-orientador} % no caso de 2 co-orientadores, usar esta sintaxe
%\coorientadorb{Nome do Co-orientador 2}	% este comando inclui o nome do 2o co-orientador
\coorientador{Prof. Dr. Pedro Luiz de Paula Filho}


\local{Medianeira} % cidade
\data{\the\year} % ano automatico

% desativa hifenizacao mantendo o texto justificado.
% thanks to Emilio C. G. Wille
\tolerance=1
\emergencystretch=\maxdimen
\hyphenpenalty=10000
\hbadness=10000
\sloppy




% informações do PDF
\makeatletter
\hypersetup{
	%pagebackref=true,
	pdftitle={\@title}, 
	pdfauthor={\@author},
	%pdfsubject={\imprimirpreambulo},
	pdfcreator={LaTeX with abnTeX2},
	pdfkeywords={}{abnt}{latex}{abntex}{abntex2}{relatório técnico}, 
	%	colorlinks=true,       		% false: boxed links; true: colored links
	%	linkcolor=blue,          	% color of internal links
	%	citecolor=blue,        		% color of links to bibliography
	%	filecolor=magenta,      		% color of file links
	%	urlcolor=blue,
	%	bookmarksdepth=4
}
\makeatother
% --- 


%---------- Inicio do Documento ----------
\begin{document}


\capa % geracao automatica da capa

%\folhaderosto % geracao automatica da folha de rosto

\termodeaprovacao

% Lembre-se de que a ficha catalografica eh impressa no verso da folha de rosto
% Ficha catalografica
\fichacatpum{T137}
\fichacatautor{Sobrenome, Nome}
\fichacatpgbib{\pageref{bibstart}-\pageref{bibend}}
\fichacatpalcha{1. Teoria do controle. 2. Redes de comutação. 3. TCP/IP (Protocolo de rede de computação), ...}
\fichacatpdois{CDD (22. ed.) 621.3}
\fichacatbib{Biblioteca xxxxxx}
%\fichacat

% insercao da ATA
%\includepdf{ata.pdf}


% dedicatoria
%\begin{dedicatoria}
%Texto da dedicat\'oria.
%\end{dedicatoria}

% agradecimentos (opcional)
%\begin{agradecimentos}
%Texto dos agradecimentos.
%\end{agradecimentos}

% epigrafe (opcional)
%\begin{epigrafe}
%Texto da ep\'igrafe.
%\end{epigrafe}

%resumo
\begin{resumo} 
	%	Máx 500 palavras
	
		O cultivo da soja \lw{Glycine max (L.) Merr.} representa grande importância para o Brasil, por tanto, laboratórios de análise de sementes realizam diversos testes para mensurar as condições de qualidade dos lotes produzidos. Dentre estes testes, destaca-se o teste de tetrazólio que classifica e avalia o vigor e viabilidade de plantio, porém, a etapa de avaliação visual do teste pode gerar subjetividade em alguns casos além de ser cansativa e tediosa para os analistas. 
		O uso de visão computacional para auxiliar a identificação de padrões nas sementes tem impulsionado pesquisas com diversas técnicas em processamento de imagem e diferentes classificadores de dados. Nota-se que a maioria das pesquisas utiliza conjuntos distintos de imagens e não os disponibilizam para que comunidade científica, dificultando a reprodução dos resultados e a comparação com outros métodos sobre os mesmos dados.
		A proposta deste trabalho é elaborar uma metodologia de coleta de dados e desenvolvimento de uma base de imagens publicas, que fomentará pesquisas cientificas no âmbito do teste de tetrazólio em sementes de soja. 
		A metodologia exige uma abordagem simples na etapa de coleta de dados permitindo a aquisição da imagem e a classificação da amostra, alterando o mínimo possível o tempo final da análise quando comparada ao método tradicional. A concordância dos dados coletados devem ser verificados entre os pares, onde uma reavaliação é feita por outros especialistas sobre uma imagem já classificada, então é calculado o coeficiente estatístico de concordância kappa e AC1 dos dados.
		A implementação da metodologia proposta foi realizada com o desenvolvimento de uma ferramenta de coleta e uma plataforma de disponibilização dos dados. Para ferramenta de coleta de dados é utilizado um \ew{software} para dispositivo móvel, como \ew{smartfones} e \ew{tablets} e um sistema(web) para publicização em um banco de dados de imagens, oferecendo uma base verificada e segmentada de dados e imagens de sementes de soja que passaram pelo teste de tetrazólio.
		
		%sistemas de auxílio a agricultura (\ew{CAA - Computer-Aided Agriculture}).
		
			 
	\end{resumo}
		

%abstract
\input{abstract.tex}

% listas (opcionais, mas recomenda-se a partir de 5 elementos)
%\listadefiguras % geracao automatica da lista de figuras
%\listadetabelas % geracao automatica da lista de tabelas
%\listadequadros % adivinhe :)
\listadesiglas % geracao automatica da lista de siglas
%\listadesimbolos % geracao automatica da lista de simbolos

% sumario
\sumario % geracao automatica do sumario

%\setcounter{page}{12}

%---------- Inicio do Texto ----------
% ELEMENTOS TEXTUAIS 
% Apresentam a exposição do conteúdo efetivo do trabalho. Um trabalho acadêmico possui três partes fundamentais: introdução, desenvolvimento e conclusão.
%
% Introdução
% Parte inicial do texto, na qual devem constar o tema e a delimitação do assunto tratado, objetivos da pesquisa e outros elementos necessários para situar o tema do trabalho, tais como: justificativa, procedimentos metodológicos (classificação inicial), embasamento teórico (principais bases sintetizadas) e estrutura do trabalho, tratados de forma sucinta. Recursos utilizados e cronograma são incluídos quando necessário
%
% Desenvolvimento
% Parte principal do texto, que contém a exposição ordenada e pormenorizada do assunto. É composta de revisão de literatura, dividida em seções e subseções, material e método(s) e/ou metodologia e resultados, agora descritos detalhadamente. 
% Cada seção ou subseção deverá ter um título apropriado ao conteúdo.
% Deve-se utilizar sempre a terceira pessoa do singular na elaboração do texto, mantendo-se a forma impessoal no mesmo.

% Conclusão
%Parte final do texto, na qual se apresentam as conclusões do trabalho acadêmico, usualmente denominada Considerações Finais. Pode ser usada outra denominação similar que indique a conclusão do trabalho.


% recomenda-se a escrita de cada capitulo em um arquivo texto separado (exemplo: intro.tex, fund.tex, exper.tex, concl.tex, etc.) e a posterior inclusao dos mesmos no mestre do documento utilizando o comando \input{}, da seguinte forma:

%Toda pergunta precisa de uma resposta, toda resposta precisa de uma fundamentação.



O que é o seu projeto?%perguntas

Qual é o objetivo principal do seu projeto? % o que?


O objetivo é criar uma base
Uma base de informações

Uma base de informações catalográficas
Uma base de informações e imagens

de sementes de soja
Aplicadas/Expostas

ao teste de tetrazólio
ou testadas/analisadas pelo teste de tetrazólio


\emph{Mas a ideia é a Criação de uma base de informações De sementes de soja passadas pelo teste de tetrazólio para apoio a pesquisa}


-> Esse é o seu objetivo: fazer esta base %respostas

Qual é a justificativa para isso? %perguntas
Por que fazer este trabalho?

%lembrei do canvas


A justificativa é muito simples! %respostas


Não existe nenhuma base hoje conhecida sobre sementes e informações das sementes que passaram pelo teste de tetrazólio principalmente contendo imagens % por que

Não tem isso hoje
tudo bem, não tem, mas ...

por que precisa ter? %para que


Por que vários trabalhos são feitos de pesquisa em cima do tetrazólio % apoio a pesquisa

% para resolver um problema
E hoje, vários trabalhos na área de computação estão sendo feito principalmente na área de processamento de imagens
porém quando o pesquisador vai desenvolver algum trabalho
neste sentido ele desenvolve sua própria base de dados
Ele captura sua própria imagens faz a catalogação das suas imagens e faz o estudo


% os problemas
Aí eu consigo ver pelo  menos dois problemas. Pelo menos dois

O primeiro deles: % apelo a um pilar da pesquisa
Se o pesquisador não publica torna publico
Se ele não torna publico o que ele utilizou
Fica muito difícil a reprodução do trabalho
que é muito importante na área de pesquisa
para a pesquisa ser reproduzida


% apelo ao custo da pesquisa
outro problema menos grave % menospresa um problema mesmo grande, para elevar ainda mais o outro
mas também muito importante
é o tempo
não vou usar só o tempo, o custo também
tempo e custo
custo em dinheiro, custo em tempo
custo em esforço
Recursos %apelo aos prazos apertados e baixos financiamentos
%criar a identificação do leitor pesquisador com o problema


para aquisição das imagens
ou se você não tem uma imagem
Não tem uma base
e precisa ter, você tem esse custo a mais

são estes dois motivos: dificuldade de reprodução e custo %Promessa de resolver dois problemas com uma unica soluçao

%aplicação da ideia a um caso específico
essa base precisa ser criada para apoiar futuras pesquisas na área
de tecnologias em sementes principalmente sementes de soja
especificamente usando o teste de tetrazólio
%convenver que a area de aplicação é interessante
%gosto particular
%área importante ao Brasil. Citar o capitulo da dissertação sobre o soja

como fazer isso? %mais perguntas 

primeiro vamos falar de quem já fez e como fez % mais respostas

você vai ter que procurar e citar %embasamento para responder
Vários trabalhos - {\LARGE Isso é importante, vá atrás disso} -
vários trabalhos que fizeram análise de imagem em sementes
de tetrazólio
reconhecimento de padrões
se quiser deixar mais amplo
não fique só no soja
quer ficar mais amplo: não fique só no tetrazólio
talvez seja bom para contextualizar a quantidade de pesquisa feita
com processamento de imagens nesta área para esse publico, para essa área de pesquisa
%ampliou a area de aplicação
%ajuda a justificar a ideia, e expor as possíbilidades onde o projeto pode chegar

% mas já coloca os pé no chão e limita o escopo, porém deixando a
%sensação que trabalhos diferentes podem usar a mesma estratégia
%desperte a atenção exibindo uma oportunidade fácil de trabalhs futuros
tem que fechar isso aí, deixar claro que é só soja e tetrazólio, para não ficar muito amplo
então você tem que procurar esses projetos e esses trabalhos
para amostrar quais bases eles usaram
se a gente conseguir traçar/encontrar
uma base comum que eles esteja usando, a gente referencia

%trabalhos anteriores parecidos e um pouco de informações técnicas
então, projetos que não usam tetrazólio, usam a base X para essa pesquisa
É importante para gente saber a característica desta base
que informações ela tem sobre a semente
e quantas imagem, e formatos, as dimensões da imagem outro ponto
fechando, de tetrazólio como foi feito

Não achando as bases:
Os artigos descrevem suas bases particular
quantas imagens acharam
Quantas de cada classes

Agora a gente procura a maior base, o maior trabalho, sei lá usou mil imagens
Fez o processamento de imagem
procurando aí, sei lá, 4, 5 classes
vamos por 4, 250 imagens por dano
é um numero que dependendo do trabalho, é ok
ou se for 100 por classes, teremos 4000 imagens

cita e fala que é o maior trabalho
informações, por exemplo, qual é o dano mais aparente

Onde está o dano
está segmentado ou não

qual é a espécie, qual é a safra, qual é a região em que foi plantada?
qual é o período em que foi colhida, qual é o peso de mil gramas?
qual é o lote e o resultado final?
para que essas informações? Na minha base tem que ter.

Não vai apoiar pesquisa só para processamento de imagens
apesar de ser o foco
Se eu quiser fazer uma análise estatística
Mostra o tamanho da base e fala
Bom se uma base de 4000 com essas informações
Com uma técnica mais recente de processamento de imagens
necessitaria de muito mais imagens
Isso são para processamentos convencionais, todos esses projetos usaram isso
Outros projetos que demandam muito mais imagens
necessitam de em média de 10.000 imagens por classe
E a gente tem que citar trabalhos que usem essa quantidade de imagens
então já vou
Quando eu falar de importância de base, vamos falar sobre o MNist
que é uma base muito importante
falar o quanto ela foi fundamental e foi usada disseminadamente em vários projetos
O NonMnist dá para falar também
Talvez quando falar de NonMnist, dê para puxar o DeepLearn
que usa muitas imagens
então uma técnica nova
que demanda muita imagem
sua base vai ser
Não existe uma base ainda
deste tamanho, para está aplicação
e você vai fazer, esse é o seu objetivo
Não fazer só uma base, mas fazer uma base grande, muito boa,
completa. Que apoie a pesquisa .
Justamente por que você quer desenvolver um projeto
que classifique a sua base usando deepleaning
Esse era o seu primeiro objetivo
agora você constrói a base, depois você constrói o classificador
fechou
Depois você pede para outro você fazer isso mais pra frente
pro enquanto você vai fazer a base
depois que fizer a base você fala com o próximo
Baseado em projetos
vamos estimas que a gente queira 10 mil imagens
10 mil imagens
por classe, umas 40 mil imagens, 4 classes 40 mil imagens
8 mil imagens
8 mil não, 80 mil imagens
Você vai falar para banca que vai fazer uma base de 80 mil imagens
Fala que vai fazer uma base de 160 mil imagens, o dobro
160 mil imagens
agora tem que calcular o tamanho dessa base
tudo bem, é isso que precisa uma base bem grande
Daí vamos pensar que vaí tirar estas imagens
Se eu fiquei 3 dias para tirar 90, quanto tempo vai precisar?
O negócio é que eu já criei uma forma de coletar
Vamos dividir, vamos colocar muita gente coletando as imagens
Muita gente tirando essas fotos
A estratégia é o seguinte
Quando eu coletei as imagens
Usei uma máquina macro com lentes, estudio
Iluminação controlada, sombra controlada
fiz um procedimento normal e não tive muito sucesso
Não por que as imagens não estavam padronizadas
mas por que tinha pouca imagem
Então a padronização é importante
mas a quantidade também é
E a quantidade é muito mais importante que a padronização dependendo da tecnica
Daí eu preciso que você ache esses dois trabalhos, é importantíssimo que você ache isso aí
existem
Dois, muitos muitos trabalhos. Ache dois e mande
vários trabalhos usando DeepLearning
que você tira uma foto de uma flor
procure no seu quintal uma flor
tire uma foto
e mande para rede neural, deep learning
e ela vai disser qual a espécie daquela flor
tem que treinar com muitas, centenas de imagens
milhares de imagens
para identificar vários tipos de flores
essas fotos não foram padronizadas
justamnete para poder identificar
e determinar esses indicadores
determinar o que realmente indica
a espécie
e isso é o que muda
É uma grande diferença do processamento de imagem tradicional
Onde você implementa vários métodos de extração de características
depois uma rede neural artificial poder classificar
passa por uma rede neural classificador
Ou o deeplearning que você passa a imagem e deixa ela extrair
ela procurar as características
e nisso como é você que faz o trabalho de procura
toda parte
de programação
que você teria para extração das características
não é necessária, a maquina vau fazer isso, deixa ela trabalhar
e aí
pessoas tirando fotos
nos laboratórios hoje
E os analistas enquanto fazem a análise vão tirar fotos dos celulares deles para você
já dá pra ver muito problema aí, então tem que começar a controlar um pouco
primeiro
primeiro que não é tão simples assim, só tirar a foto
eu quero garantir
E você quer também
um pouco de padronização
então pensei em uma forma bem legal
a pessoa vai tirar uma foto
vamos imaginar o seguinte
quanto a pessoa vier tirar
embaixo aqui, tem que ter um gabarito
mas você sabe
eu já sei o que é isso e você sabe
a gente já sabe disso a bastante tempo
quando a gente pensou nisso aqui
estava começando
mas eu vou falar tá
tem que ser um quadrado
um retangulo
vamos fazer em acrílico
recortar em acrílico, vai ter que desenhar isso aí
desenhar, recortar
despachar para todos os laboratórios
talvez você faça algo automontáveis
que você manda pelo correio e a pessoa pega as peças de acrílico, destaca e monta
Próxima parte
Essa caixinha, tem que ter uma tampa transparente e dentro dois espelhos em V
e um espelho assim
90 graus
o espelho
a tampinha em cima
quando a gente coloca a semente aqui em cima
com a base dela aqui
metade metade
colocou aqui virada para baixo
a imagem do fundo dela
vai refletir no fundo dela
que vai refletir neste espelho
que vai refletir para cima, deste lado
então colocou a semente aqui
com a parte de baixo dela
vai bater no espelho
que vai estar na diagonal
vai angular pra ca 90
angulou aqui, sobe
se olhar de cima
vai olhar a semente que você colocou
um semente aqui e vc vê as costa dela
e aqui por espelhos vê o interior dela
e é por aqui que você vai tirar a foto
é só vir com o celular e tirar aqui
resolveu o problema do frente e verso
só que esta caixinha
nem tinha falado como problema
cada semente
tem que ser cortada
lembra na apresentação de falar
Olha só como eu vou colocar
Um monte de pessoas tirando fotos de qualquer jeito
é um problema muito grande
como é a foto
cada lote o cara tem 100 sementes para tirar fotos
o primeiro que você desenvolveu foi um método
o método para tirar essas fotos
para depois sistematizar
quando coloca aqui uma foto já tem todos os lados
evita que tire foto errada e ajuda muito na identificação
quando
você tira uma foto e depois tira outra
para saber que aquelas costas
e aquela outra imagem que está guardada
tem que fazer um alinhamento
tem que pegar o contorno, ver a geometria, o poligono e ver se encaixa
nenhuma eu acho que vai estar
vai precisar rotacionar
pra encaixar certinho
mas esse problema vai acabar
tirar uma foto com a frente e as costas certinha
uma torta a outra torta certinha, igual no mesmo angulo
facilita
muito
e se a pessoa
tirar a foto com o celular, meio assim, assim, assim
então neste seu gabarito
nos 4 cantos vamos colocar marcadores
então por exemplo, vai ser pintado de preto todos os cantos
por der meio centímetro por meio centímetro
dois milímetros por dois milímetros
quando a pessoa for tirar uma foto
e tiver angulado
pelo tamanho, que ficou a marcação em relação a outra
você já sabe se foi ou não foi
se imagem está torta ou não
de qualquer lado que você ver, não precisa nem ter giroscópio
se tiver reto pelo processamento de imagem
depois que a pessoa tira a foto, ela tem que mandar para mim
os danos que a semente tem
e me enviar tudo organizado
160 mil fotos
mas olha, vamos riscar dois zeros por que são 100 por amostras
1.600 amostras
num dia a pessoa faça
umas 5 só
só cinco
160 dias, isso uma pessoa
160mil fotos
tira 2 zeros, 1.600 amostras
agora eu quero saber quantas amostras uma pessoa faz por dia, 5. Então divide por 5
1.600 / 5
320 eu acho
320
320 amostras
só? 320 amostras
não. Você precisa de 1.600 amostras
vai precisa de 320 dias
ou em um dia 32o pessoas
em um dia só
Vamos dizer que você não consiga 320 pessoas
mas você consegue em 10 dias com 32 pessoas
em 10 dias voc? tem a base que você precisa
mas agora imagina 32 pessoas mandando toda essa informação
tem que mandar, tem que organizar, tem que receber certinho, pessoal não pode
não pode atrapalhar o processo dela
e parar para tirar uma foto e mandar tudo isso vai diminuir o rendimento da pessoa
Já sei que é possível fazer
Agora qual é a estratégia para que as pessoas queiram fazer isso por você
essa conta rápida foi para provar que dá
agora eu preciso
desenvolver um processo
para que o analista possa usar, esse processo não pode atrapalhar ele
não pode
não pode tomar mais tempo do que já é gasto
a análise não pode demorar mais do que já demora
senão ele não vai pode fazer
Mas se você está aplicando mais tarefas para fazer
e não quer que aumente o tempo, então
você tem que baixar o tempo das outras tarefas que ele tem
a tarefa que ele tem hoje é
no teste de tetrazólio
a sua estratégia consiste em melhorar o processo original
Em baixar o tempo original, para que
o que ele faz hoje, ele consiga fazer mais rápido
ele vai ter tempo de fazer o dele e mais o seu
e vai poder te ajudar, senão não vai fazer
primeiro que para o cara enviar isso
se demandar de tirar a foto, organizar, planilhar, mandar por email, ou para servidor, usar site
esquece por que não vai funcionar, você vai ter que dar treinamento
vai ter que organizar, vai vir uma base totalmente
você vai fazer um aplicativo de celular que vai usar a câmera do celular e transferir do celular
do aplicativo e já bate as fotos e te envia
colhe os dados, cadastra e manda, vai chegar para você
mas como esse aplicativo vai diminuir o tempo
toda solução dentro de uma aplicativo, fica facil de atualizar
você vai trabalhar com um S.O. só
é suficiente e você sabe disso
o processo deles hoje é bem interessante
eles pegam as 50 sementes e cortam ao meio
as 50 sementes, eles colocam numa solução
de sal de tretrazólio
para agir com a engima do soja
se ela ficar vermelha tem sua análise visual feita pelo técnico treinado
eles cortam cada semente ao meio e analisam o conjunto
quando eles fazer isso eles pegam uma ficha de papel
uma caneta
e tem 50 linhas
50 linhas para cada tipo
por exemplo
dano de umidade, dano mecânico, dano de percevejo
coloca isso na sua apresentação, tem lá no livro do França
então eles vão colocar um simbolozinho
tipo um corte quando um tipo, uma linha quando outro
um tracinho, um X
eles fazem toda análise e vão marcando em uma folha de papel
]no final eles contam
quantos tem de cada, anotam o subtotal
fazem uma média de cada um e dão os resultados
Então este aplicativo vai substítuir este papel
então a pessoa tira uma foto
primeiro captura, depois classifica
próximo
segundo, tira foto, dano tal
e põe alia
ele não vai mais indicar isso escrevendo no papel
ele não vai mais perder o tempo escrevendo no papel, vai usar o tempo para escrever no aplicativo
No final, que ele terminou a analise, você tem os dados que precisa
e eles não tem o relatório dele
tem que ter
então você vai ter que gerar o relatório para ele
a mesma coisa que ele faz, seu aplicativo vai precisar fazer
Qual a vantagem?
você pode far para ele a versão igual a que ele tem
então ele tem o que precisa
nessa estratégia, você consegue que as 3
os 3
interessados, se deem bem
tenham vantagem
um ganha ganha
primeiro ganha o analista
que o processo dele fica mais fácil e mais rápido
usando o aplicativo
Não precisa fazer conta
O relatório está lá pronto para ser impresso e assinado
se precisar imprimir
a empresa ganha
por que a empresa que tem laboratório de sementes
terão documentado de forma digital
Não só em PDF do relatório
que eles não tinham isso
vamo começar a entregar um pouco de beneficio
gerar valor
e depois, eles precisam lançar no sistema interno
tem que registrar isso
toda empresa tem
nosso sistema poderá entregar esses dados formatados, tabulados, em xml idependente do formato
vai pegar um xml e vai entregar
então você recebe as fotos que não tinha antes, que servem para uma auditoria
recebe o relatório digital
que pede ser exatamente como o atual ou com foto
e um xml, que ele pode importar para o sistema dele que ele ganha velocidade
Então o cara usando esse aplicativo a empresa ganha tempo
auditoria
armazenamento de dados, velocidade, integridade, muitas coisas e menos trabalho para ele
que não vai precisar calcular o relatório
que mais?
eu ganho, por que eles vão coletar para mim as informações
recebe um pacote organizado
o pulo do gato está aqui
recebi um monte de imagem e um monte de dados
como eu falo uma base validada para pesquisa?
bom se eu tenho uma classificação e tenho uma imagem
eu posso tentar validar
se está informação é coerente ou não
eu posso pegar no mesmo aplicativo emm outra area
de verificação
que pegue imagens que outra pessoa tirou e envie para aquele analista e pergunte, classifique para mim isso aqui
e vou colocar a classificação dele

%\chapter{Introdução}
\section{Objetivos}
\section{Justificativa e contribuições}
\section{Organização do trabalho}

\chapter{Revisão de Literatura}
\section{A reprodutibilidade da pesquisa científica}
\section{Fontes de dados}
\subsection{Fontes de dados para pesquisa com imagens}
\subsection{Catalogação de dados}
\subsection{Estatísticas}

\section{A Soja}
\subsection{A importância da soja para o mercado financeiro e na indústria alimentícia}
\subsection{A importância da soja para o Brasil}
\subsection{Sobre o controle de produção}
\subsection{Sobre os laboratórios de sementes e os testes}
\subsection{Sobre o tetrazólio}

\section{Flutter}
\subsection{Histórico}
\subsection{Beneficios}
\subsection{Dart}
\subsection{Material Designeer}


\chapter{Material e Métodos}
\section{Captura de imagens}
\section{Analise especialista}
\section{Validação da análise}
\section{Estratégia de uso}




\begin{list}{label}{spacing}
	\item 
\end{list}


\chapter{Introdução} \label{ch:intro}

O cultivo da soja \lw{Glycine max (L.) Merr.} representa grande importância para o Brasil, promovendo desenvolvimento tecnológico, movimentando o mercado financeiro e a industria alimentícia. Ainda, interioriza o processo de urbanização, gerando emprego, renda e arrecadação de impostos.

Para garantir que o processo de cultivo seja eficiente, laboratórios de analise de sementes realizam diversos testes para mensurar as condições de qualidade dos lotes produzidos. Para avaliação do vigor de sementes de soja, recomenda-se os de envelhecimento acelerado, tetrazólio, condutividade elétrica, crescimento de plântulas, classificação do vigor de plântulas (Vieira et al., 2003). Destes, o teste de tetrazólio se destaca, principalmente para a soja, devido à sua rapidez, precisão e também pelo grande número de informações fornecidas pelo mesmo \cite{FrancaNeto1998}.

Na etapa de interpretação do teste de tetrazólio, o analista desfruta de seus conhecimentos da fisiologia de sementes, para classificar visualmente cada uma das cem sementes por lote individualmente. \citeonline{FrancaNeto1998} aponta que a precisão dos resultados depende diretamente da experiência, interpretação dos dados e julgamento crítico do analista. E por ser relativamente tedioso, uma vez que as sementes são avaliadas uma a uma, não é recomendado a realização de muitos testes no mesmo dia, para que o cansaço não atrapalhe o resultado da análise.

Em atividades que podem ser demoradas e estressantes, o uso de sistemas computacionais inteligentes pode automatizar processos proporciona resultados resultados melhores e mais confiáveis que o esperado para o ser humano (Lesk, 2008). Os sistemas que utilizam técnicas de Visão Computacional, são exemplos de sistemas que buscam assemelhar-se à visão humana para solução de problemas complexos como o reconhecimento de imagens.

Pesquisas como \citeonline{MECHELN} e \citeonline{MarcosFilho2009}, propõe o desenvolvimento de metodologias automáticas para realização da classificação de sementes de soja pelo teste de tetrazólio, utilizam técnicas de Visão Computacional. 

Como exposto, a realização de pesquisas para o desenvolvimento de sistemas computacionais para classificação de sementes, especialmente de tetrazólio, requerem fundamentalmente o uso de conjuntos de dados mais robusto e confiável. 
Diante deste contexto, a presente pesquisa propõe o desenvolvimento de metodologia de catalogação de imagens e captura de informações \lw{In loco} para uso em laboratórios de classificação de sementes.


\section{Objetivos}

Criar uma base catalográfica de imagens baseada em visão computacional, com informações associadas, de sementes de soja submetidas ao teste tetrazólio.


\begin{enumerate}
	\item Desenvolver um aplicativo para dispositivos móveis para coleta das imagens e classificação;
	\item Sincronizar as informações com um servidor centralizado;
	\item Gerar relatório para análise estatística dos dados;
	\item Gerar \ew{data sets} das informações;
\end{enumerate}

\section{Justificativa e contribuições}

Com a intenção de possibilitar trabalhos futuros em reconhecimento de padrões com visão computacional, que identifiquem os danos em sementes de soja auxiliando no processo de análise do teste de tetrazólio, este trabalho tem o objetivo de projetar, coletar e disponibilizar uma base de informações com imagens ampla e confiável.

A metodologia de desenvolvimento foi projetada para que o analista profissional possa fazer a coleta dos dados durante sua atividade de rotina sem prejudicar seu desempenho habitual. Além de fornecer uma ferramenta que continuará auxiliando-o mesmo após a finalização deste projeto de catalogação.

Esta metodologia promove benefícios diretos e imediatos aos envolvidos no processo de execução do teste de tetrazólio em sementes de soja: ao analista de sementes, ao laboratório responsável, a empresa que comercializa a as sementes, aos órgãos fiscalizadores e ao produtor que adquire o lote de semente. Ao desenvolvê-lá desta forma abriu-se as possibilidades de parcerias com laboratórios de sementes, na colaboração com equipamentos e mão de obra qualificada.

%Esta metodologia é descrita detalhadamente no capitulo \ref{ch:meto}.


\section{Organiza��o do trabalho}


A organiza��o da estrutura deste trabalho encontra-se dividida em seis cap�tulos, al�m das refer�ncias bibliogr�ficas. Neste primeiro cap�tulo � feita uma introdu��o do trabalho apresentando-se os objetivos a cumprir. 


No segundo cap�tulo s�o apresentados conceitos b�sicos sobre o teste de tetraz�lio e os principais danos detectados nas semente durante a an�lise.

O terceiro cap�tulo faz uma revis�o dos conceitos relacionados a sistemas CBIR e � literatura existente na �rea, descrevendo as t�cnicas mais utilizadas considerando sua rela��o com a �rea m�dica. 

O quarto cap�tulo apresenta, de forma detalhada, a metodologia proposta para modelagem do banco de dados e no quinto cap�tulo � descrito a metodologia de CBIR proposto. 

O sexto cap�tulo trata da apresenta��o de resultados da metodologia proposta, para tal fim s�o utilizadas imagens t�rmicas e mamografias. 
O �ltimo cap�tulo faz considera��es finais e prop�e id�ias para trabalhos futuros

\chapter{Revis�o de Literatura}
\label{ch:fund}


\section{A reprodutibilidade da pesquisa cient�fica}
\label{sec:reprod}

A confiabilidade e a reprodutibilidade s�o pilares da pesquisa cient�fica moderna, garantindo que os resultados produzidos possam promover o avan�o do conhecimento. O m�todo cient�fico na ci�ncia moderna, criada por Francis Bacon, postula que devemos formular uma hip�tese, desenhar experimentos para confirm�-la ou refut�-la, analisar os resultados de maneira imparcial, inserir esses resultados no contexto do conhecimento atual e reproduzir esses experimentos.

Entretanto a comunidade cient�fica j� identificou e assumiu, em artigos em revistas especializadas, que pesquisadores t�m dificuldade em reproduzir experimentos de outros pesquisadores. Principalmente depois que as empresas farmac�uticas Bayer e Amgen terem reportado n�o poder reproduzir grande parte dos artigos que descrevem drogas como potenciais novos f�rmacos para o tratamento de c�ncer e outras doen�as.

%https://blog.scielo.org/blog/2017/02/08/avaliacao-sobre-a-reprodutibilidade-de-resultados-de-pesquisa-traz-mais-perguntas-que-respostas
%https://blog.scielo.org/blog/2016/03/31/reprodutibilidade-em-resultados-de-pesquisa-os-desafios-da-atribuicao-de-confiabilidade
%https://blog.scielo.org/blog/2013/07/31/artigo-propoe-quatro-pilares-para-a-comunicacao-cientifica-para-favorecer-a-velocidade-e-a-qualidade-da-ciencia
%http://www2.fesbe.org.br/reprodutibilidade-na-ciencia-uma-noticia-ruim-e-varias-boas/

Iniciativas de reprodutibilidade come�aram a surgir em ag�ncias de divulga��o cient�fica como a \ew{Science Exchange}, que desenvolveu em 2013 a \ew{"Reproducibility Project: Cancer Biology"}, com objetivo de validar 50 artigos publicados entre os mais relevantes e de alto impacto na pesquisa oncol�gica e \ew{"Reproducibility Project: Psychology"} com a proposta de avaliar a reprodutibilidade de 100 artigos de pesquisa em psicologia, iniciado em 2011 e conclu�do em 2015, e movido por den�ncias de fraude e an�lise estat�stica falha em estudos cl�ssicos de psicologia.


Uma pesquisa online realizada pela \ew{Nature} com mais de 1.500 pesquisadores de todas as �reas do conhecimento e publicada em 2016 relevou que mais de 70\% n�o teve sucesso ao tentar reproduzir experimentos de terceiros e mais de 50\% n�o pode reproduzir seus pr�prios experimentos. No entanto, apenas 20\% dos entrevistados afirmam ter sido contatados por outros pesquisadores que n�o puderam reproduzir seus resultados. Este t�pico, entretanto, � delicado, pois corre-se o risco de parecer incompetente ou acusat�rio. Pelo contr�rio, quando um resultado n�o pode ser reproduzido, a tend�ncia dos cientistas � assumir que existe uma raz�o perfeitamente plaus�vel para o insucesso. De fato, 73\% dos entrevistados s�o de opini�o que ao menos 50\% dos resultados em suas �reas s�o reprodut�veis, sendo os f�sicos e qu�micos entre os mais confiantes. (rescrever)

Quanto � causa da irreprodutibilidade, os fatores mais comuns est�o relacionados com a intensa competi��o e press�o por publicar. Entre os motivos mais citados pelos pesquisadores est�o: publica��o seletiva de resultados; press�o por publicar; baixa signific�ncia estat�stica; n�mero insuficiente de repeti��es no pr�prio laborat�rio; supervis�o insuficiente; metodologia indispon�vel; design experimental inadequado; dados-fonte indispon�veis; fraude; e avalia��o por pares insuficiente.


\section{Fontes de dados}

%https://publicient.hypotheses.org/525
%https://publicient.hypotheses.org/425
%https://blog.scielo.org/blog/2014/07/14/movimento-open-data-se-consolida-internacionalmente/#.W4F5wq1jucM
%https://www.google.com/search?client=firefox-b-ab&ei=SlGBW4vME4T7wQShrbzQCw&q=datasets+para+pesquisa+cient%C3%ADfica&oq=datasets+para+pesquisa+cient%C3%ADfica&gs_l=psy-ab.3...25338.27160.0.28160.0.0.0.0.0.0.0.0..0.0....0...1.1.64.psy-ab..0.0.0....0.vo-Mgdfl8qo


Em artigos cient�ficos tradicionais os dados s�o vistos como um apoio para as conclus�es do artigo de pesquisa, em publica��es de dados principalmente em boletins s�o apresentados novos dados com o objetivo de apoiar novas pesquisas.
A \ew{United States Department of Agriculture} \sigla{USDA}{\ew{United States Department of Agriculture}} atualiza a comunidade cient�fica frequentemente com publica��es dados sobre a produ��o, fornecimento e distribui��o agr�cola mundial e 11 dos 20 artigos mais citados das revista \ew{Bulletin of the American Meteorological Society} s�o sobre dados.

Para muitos jornais esse tipo de \ew{data paper} recebe a mesma avalia��o do que os outros artigos de pesquisa, sem instru��es especificas. Outros jornais possuem instru��es sobre a disponibiliza��o dos dados para avalia��o por cientistas, por�m alguns barreiras como a falta de colaboradores e a avalia��o de dados por pares ainda com defini��es n�o t�o claras, criam uma realidade diferente da esperada. 

A preocupa��o de manter a publica��o permanente dos dados e o vinculo com artigos que os citam, incentivam a parcerias entre bibliotecas e jornais com ferramenta de reposit�rios de dados (por exemplo http://rda.ucar.edu), de \ew{persistente identifier} como o \sigla{DOI}{\ew{Digital Object Identifier}} e da homogeniza��o dos metadados. Esses v�nculos entre reposit�rios e jornais est�o crescendo r�pido e os atores formam uma comunidade de avalia��o de dados.

Os reposit�rios de dados criados para institui��es ou por centros nacionais de dados, tem uma avalia��o mais t�cnica do que cientifica. Mas a situa��o esta evoluindo e alguns reposit�rios come�am a exigir uma avalia��o do conjunto de dados antes de public�-los, como a implanta��o de um plano de gerenciamento de dados que exponha antes do deposito a qualidade e controle dos dados.

\emph{Data Jornals} como Earth System Science Data (Copernicus), Geoscience, Data Journal (Wiley), e Scientific Data (Nature), s�o jornais espec�ficos para publica��o de \ew{data paper} que descrevem um conjunto de dados, disponibilizado dentro de um reposit�rio, explicando as condi��es de processamento, a cole��o e formato de arquivos.


\subsection{Fontes de dados para pesquisa com imagens}

� comum na �rea de pesquisa em processamento digital de imagens o estudo de fontes de dados tradicionalmente utilizadas na literatura cientifica, quando se prop�e novos m�todos de an�lise, podendo comparar os resultados obtidos com trabalhos anteriores. Com exemplo o projeto MNIST.

O projeto MNIST publicado em 1995 � banco de dados de d�gitos manuscritos, dispon�vel em http://yann.lecun.com/exdb/mnist/, possui um conjunto de 70.000 exemplos de d�gitos normalizados por tamanho e centralizados em uma imagem de dimens�es fixas. � um banco de dados para pesquisadores que querem experimentar t�cnicas de aprendizado e m�todos de reconhecimento de padr�es em dados do mundo real, enquanto gastam esfor�os m�nimos em pr�-processamento e formata��o.

Algumas t�cnicas de reconhecimento de padr�es pro processamento de imagens exigem um elevado numero de amostras. Neste sentido surgem projetos como notMNIST de 2011, com a proposta mais de 500 mil imagens de letras geradas digitalmente em fontes tipogr�ficas diferentes. Este projeto apoia a pesquisas em reconhecimento de textos digitalizados ou \sigla{OCR}{\ew{Optical Character Recognition}} e e reconhecimento de textos para tradu��o em tempo real usando a c�mera de celulares.

\begin{figure}[htb]
	\label{teste}
	\centering
	\begin{minipage}{0.4\textwidth}
		\centering
		\caption{Exemplo MNIST} 
		\includegraphics[scale=0.3]{img/mnist.jpeg}
	\end{minipage}
	\hfill
	\begin{minipage}{0.4\textwidth}
		\centering
		\caption{Exemplo notMNIST} 
		\includegraphics[scale=0.3]{img/nmn.png}
	\end{minipage}
\end{figure}

\begin{figure}[htb]
	\caption{Tradu��o de textos em imagens}
	\begin{center}
		\includegraphics[scale=0.3]{img/google_tradutor.jpeg}
	\end{center}
\end{figure}


O desenvolvimento de bases de imagens para pesquisa s�o mais comuns no campo da medicina, motivados pelo desenvolvimento de tecnologias de apoio a disgn�sticos, como o projeto DMR - Database For Mastology Research (http://visual.ic.uff.br/dmi/) do Instituto de Computa��o da Universidade Federal Fluminense IC/UFF, que disponibiliza imagens mastol�gicas t�rmicas, mamografia, resson�ncia magn�tica e de ultrassom para a detec��o precoce do c�ncer de mama.


\subsection{Cataloga��o de dados}

Quando um pesquisador n�o tem dispon�vel uma fonte de dados que atenda a �rea de estudo de seu interesse, este precisa coletar as informa��es como parte de seu trabalho, para ent�o process�-las e concluir sua pesquisa.

A necessidade de cria��o de uma base de dados espec�fica, pode exigir uma grande coleta de amostras a campo em uma per�odo curto de tempo, com v�rias repeti��es, o que demandaria uma equipe e equipamentos, aumentando o tempo e o custo da pesquisa.

Em um cen�rio de pouco incentivo em pesquisa e curtos prazos na corrida por publica��o, alguns pesquisadores, ap�s investirem na cria��o de bases de informa��es adequadas para seus estudos, n�o facilitam o acesso dos dados para que outros possam pesquisar m�todos diferente, fato que poderiam gerar melhores resultados e aumentar o conhecimento cient�fico. Por outro lado, pesquisadores criam projetos que serviram de base para v�rios outros trabalhos, fomentando a tecnologia e o desenvolvimento do seu campos de pesquisa.

Este trabalho tem a miss�o de desenvolver uma fonte de dados ampla e confi�vel, para apoio a pesquisas em uma �rea importante para o agroneg�cio brasileiro: a produ��o de soja.

%a soja%
%A participação do Brasil na Produção Internacional%
%A importância da soja para o mercado financeiro e na indústria alimentícia%
%A importância da soja para o Brasil%
%Sobre o controle de produção%
%Sobre os laboratórios de sementes e os teste%
%Sobre o tetrazólio%


\section{A Soja}
A soja cultivada atualmente é muito diferente dos seus ancestrais, que eram plantas rasteiras que se desenvolviam na costa leste da Ásia, principalmente ao longo do rio Yangtse, na China.  Sua evolução começou com o aparecimento  de plantas oriundas de cruzamentos naturais entre duas espécies de soja selvagem que foram domesticadas e melhoradas por cientistas da antiga China.

Cultivada e consumida há milhares de anos pelas civilizações orientais, foi somente a partir do século vinte que foi comercialmente cultivada no Ocidente, mais precisamente nos Estados Unidos(EUA), a partir da década de 1920. Até 1940, a área de soja cultivada para forragem era maior que a cultivada para grãos. A partir de 1941, a área cultivada para grãos superou a cultivada para forragem. \cite{Amelio2011}

\subsection{A participação do Brasil na Produção Internacional}
Segundo \citeonline{FAS2018}, entre os anos de 19xx e 2017, o Brasil liderou a produção de soja no mundo ao lado dos EUA. Atualmente os dois países juntos são responsáveis por 66\% da produção mundial, com 114,10 e 116,92 milhões de toneladas métricas (mmt). A previsão da produção de soja para 2018/19 no Brasil, por \cite{FAS2018}, é um recorde de 117,0 milhões de toneladas métricas (mmt), em uma área colhida prevista para de 36,5 milhões de hectares (mha), 4\% em relação ao ano anterior, prevendo a produtividade de 3,21 toneladas por hectare, mantendo acima da média dos últimos cinco anos. 

\citeauthor{FAS2018} relata que o Brasil deverá continuar sendo o principal exportador de soja em 2018/19, impulsionados pela forte demanda global liderada pela China por alimentos proteicos, estimulando o esmagamento para obtenção do farelo o óleo de soja.

%ver \cite{Junior2017}

\subsection{A importância da soja para o mercado financeiro e na indústria alimentícia}
A commoditie de soja é negociada nas principais bolsas mundias e seu volume de negociações atualmente segundo (????) é de xxx bilhões de toneladas e yyy bilhões de dólares.\nota{Se este paragrafo for se manter, vou procurar os dados e fontes corretas} No Brasil e em vários países o valor da soja é usado para indexar contratos de diversas naturezas envolvendo o agronegócio, principalmente na compra de terras, maquinários agrícolas e empréstimos para financiamento da produção.

O grão de soja tem grande valor na economia global devido suas propriedades nutricionais e na produção de óleos
Usado na alimentação humana e animal a soja fornece grande quantidade de carboidratos e vitaminas, a proteína de soja é muito usada na indústria alimentícia e gastronômica como alternativa a proteína da carne e do leite, atendendo o mercado vegano e de restrições alimentares. A maior utilização da fibra alimentar da soja é na farinha de soja usada em raça animal, muito utilizada na produção de proteína animal, como carne bovina, suína e de frango.

O processo de beneficiamento da soja inclui a extração do óleo vegetal de soja, largamente usado na indústria alimentícia, sendo um dos mais acessíveis para o consumidor final. Segundo \cite{FAS2018a} o óleo de soja é o mais produzido e consumido no mundo entre as oleaginosas. 

\subsection{A importância da soja para o Brasil}
Para atingir tais grandezas de produção a cadeia agroindustrial da soja emprega certa de XXX pessoas no Brasil, e cerca de YYY empresas recolhendo aproximadamente XXX Milhões de reais em impostos, que representou zz\% da arrecadação do setor primário em 2017.\nota{Se este paragrafo for se manter, vou procurar os dados e fontes corretas}

O cultivo da soja no Brasil ao longo dos anos proporcionou a colonização de regiões pouco habitadas no interior do país, promovendo urbanização e crescimento econômico mais distribuído no território nacional. Entretanto cria um desafio para os produtores de soja, que encontram diferentes climas, biomas, solo, ciclos de chuvas e outros fatores, em um país de proporções continentais. Atualmente o Brasil possui registradas xxx cultivares diferentes de soja, sendo zz transgênicas cada cultivar tem suas características como variando a duração das etapas de produção, produção de óleo ou massa seca, quantidade e tamanho dos grãos, estrutura das plantas (tamanho da planta, raízes, volume de folhas), resistência a seca ou ao excesso de água, resistência a herbicidas, muitas variedades transgênicas são modificadas geneticamente para serem nocivas a determinados predadores, diminuindo o uso de inseticidas e fungicidas. Muita da tecnologia brasileira em sementes se origina da EMBRAPA, que atualmente possuem patentes de xxx cultivares de soja transgênica e zzz patentes de cruzamentos naturais.

ver \cite{Junior2017} \cite{LIMA2013} \nota{Material que comecei a ler e posso usar para embasar o texto}

\subsection{Sobre o controle de produção}
Devido a grande importância que a soja representa para nação brasileira é necessário que se proteja o processo de cultivo. Para isso o estado exige uma série de registros e controles sobre a produção, colheita, comercialização, transporte e exportação. Em cada etapa do processo produtivo da soja existem mecanismos de fiscalização por órgãos competentes.
A legislação Brasileira regula quais variedades de soja tem o cultivo permitido no Brasil, com o intuito de controlar a entrada de plantas que podem ser portas de entrada para fungos e insetos, que podem desequilibrar o ecossistema onde local. Situações como esta já aconteceram no Brasil em culturas de cacau, café e banana no passado, prejudicando a economia e o meio ambiente por vários anos

Para garantir autonomia e proteção dos custos de produção, os produtores de soja no Brasil podem produzir suas próprias sementes para safras futuras, mas se houver comercialização de sementes várias normas de qualidade devem ser seguidas, como por exemplo o teste de germinação que determina que se um lote de sementes não possuir o potencial de germinação mais que 70\% das sementes o lote é inviável para o plantio e deverá ser comercializado como grão no mercado secundário.

??? indica que uma produção inferior a 70\% de germinação é inviável em relação aos custos de produção, desta forma a utilização de lotes controlados aumentam as chances de sucesso na safra, diminuindo os riscos que as financiadoras de crédito e seguradoras rurais têm ao liberarem empréstimos aos produtores. Em 2017 xxx Milhões de reais foram liberados para produção de xx\%v das lavouras na Brasil.

\subsection{Sobre os laboratórios de sementes e os testes}
Até 2017 o Brasil possuem cerca de xx mil empresas e cooperativas que comercializam sementes de soja, sendo que xx\% possuem seus próprios laboratórios de análise de sementes, que realizam diversos testes de rotina, afim de determinar a qualidade e detectar as possíveis causas de problemas, para que estes sejam corrigidos. 

Além dos testes exigidos por lei, os testes oficiais de pureza e germinação, é comum as empresas fazer os testes de qualidade como (patologia  tetrazólio - Vigor - Envelhecimento Acelerado - Teste de emergência em areia - condutividade elétrica), para detectar patógenos e determinar inclusive se o processo de colheita e armazenagem foram feitos com qualidade ou causaram danos nas sementes e comprometeram o lote. Determinar a causa da inviabilidade de um lote é vital para ajustar o processo antes que mais lotes sejam comprometidos.

Os laboratórios de sementes empregam mais de x mil analistas no Brasil inteiro, e todo ano instituições como a EMBRAPA formam novas turmas de analistas para atenderem as demandas do mercado. O investimento que as empresas fazem com laboratório, analistas e equipamentos aumentam o custo da semente, mas agregam muito valor na qualidade e garantia das safras seguintes.

\subsection{Sobre o tetrazólio}
Uma das análises mais importantes de um laboratório de sementes de soja é o teste de \sigla{TZ}{Tetrazólio}, que a partir de uma amostra do lote de sementes, tem o objetivo de ressaltar os danos causados em cada semente, para que seja determinada em primeiro momento a viabilidade do lote e a vigorosidade que as plantas terão ao serem plantadas, e em segundo momento se houverem danos nas sementes, determinar o grau dos danos e as causas, que geralmente são causados por excesso de umidade no armazenamento, danos mecânicos do processo de colheita, secagem e estocagem; e danos causados por percevejos que se alimentam dos nutrientes contidos nas sementes.

Para realização do teste cada amostra é submetida a uma solução de sal de tetrazólio por até x horas, dependendo da metodologia usada, durante esta etapa a semente absorve a solução ficando maior e destacando em tons de vermelho carmim os danos que cada semente sofreu. Após a preparação inicial, cada semente é cortada no sentido transversal a radícula para análise do interior e exterior da mesma, na sequência é realizada uma minuciosa analise visual de cada metade da semente em busca de padrões característicos dos danos mencionados, os danos de cada semente são anotados em uma ficha de controle e ao final o analista calcula os resultados que classificam o lote.

As limitações do teste de tetrazólio, citadas por FRANÇA-NETO (1998) incluem a exigência de um treinamento especial sobre a estrutura embrionária da semente, experiência e paciência pois a análise é relativamente tediosa. Atualmente o mercado exige cada vez mais profissionais capacitados em realizar o teste.


Dada a importância que o teste de tetrazólio tem no processo de garantia de sucesso no cultivo da soja, e a importância da cadeia produtiva no Brasil e no mundo, é justificável que pesquisas sejam aplicadas com o objetivo de tornar o processo de análise mais rápido e assertivo.



%\section{Sobre o projeto}


%openCV android




\section{Flutter}


%%%%%%%%%%%%%%%%%%%

%%%%%%%%%%%%%%%%%%%

Eu vou estar indo sobre o básico de Dart, especialmente porque é usado por flutter e mais desenvolvimento de aplicativos ganhou muita tração ao longo dos anos e há um monte de lucro a ser ganho.


Eu começo falando sobre flutter e depois me movo para o dardo.



2. O que é Flutter?

Vamos começar com o primeiro take o que é flirter para aqueles de vocês que talvez não conheçam o Google Flirter como o nome indica é SDK de aplicativo móvel do Google, que pode ser usado para criar interfaces nativas de alta qualidade em iOS e Android em um curto período de tempo.

O lançamento inicial do flirter foi em maio de 2017.

Também uma nota lateral SDK significa desenvolvimento de software.

Boa.

Você deve saber que o flirter faz uso do código existente para funcionar.

Os muitos benefícios oferecidos o levaram a ser usado por desenvolvedores de aplicativos independentes e até por organizações famosas em todo o mundo.

Também é livre para usar uma fonte aberta.

Algumas das coisas básicas sobre o flerte que você deve tomar nota.

Nosso número um desenvolvimento rápido usando flirter.

Você pode ter recarregado em uma quantidade de milissegundos para garantir que seu aplicativo possa ganhar vida no menor tempo possível.

Você também pode usar uma vasta gama de widgets totalmente personalizáveis ??que ajudam a criar interfaces nativas da maneira mais rápida possível.

Eu estou falando de ser capaz de construir em minutos aqui.

Número dois interface expressiva e flexível são interface do usuário.

Seu aplicativo não pode ser bem-sucedido se não for fácil de usar para as pessoas.

Os melhores aplicativos lá fora também apresentavam a melhor experiência e a experiência do usuário.

Então, o que você conseguir mais ferramentas para garantir que o nativo e experiência do usuário é o melhor que pode ser.

A arquitetura em camadas oferece acesso à personalização completa, permitindo que você crie aplicativos interativos e amigáveis ??ao usuário, que têm uma renderização de renderização incrivelmente rápida, além de um expressivo desempenho nativo da árvore de números.

Independentemente de uma pessoa estar usando um dispositivo Android e iOS, seu aplicativo precisa oferecer a experiência que eles esperam que os widgets no flertador possam incorporar todas as diferenças críticas de plataforma para garantir a qualidade.

Essas diferenças negras incluem ícones de rolagem de navegação e garfos para a incorporação do DS às diferenças. O desempenho nativo de seus aplicativos em dispositivos Android e iOS é mantido no nível ideal.

Agora que você adquiriu um conhecimento básico do Google Flirter, deixe-me dar uma rápida introdução indireta ao escuro.



3 O que é o Dart?

O que é obscuro quando se trata de entender darte você deve saber que já é um objeto e fez e florescer a linguagem definida que você usa a sintaxe no estilo C que compila opcionalmente no javascript d'arte oferece muito apoio.

Isso significa que ele pode suportar interfaces de classes abstratas que fazem sentido digitação estática e até mesmo um sistema de tipo de som.

Agora, todos esses termos podem soar um pouco demais para aqueles que não estão familiarizados com esse jargão.

Bem, deixe-me tornar as coisas mais fáceis para você.

Basicamente escuro ajuda você a criar lindas experiências de alta qualidade em todas as telas de dispositivos por meio de uma linguagem que nega a otimização de ferramentas flexíveis fáceis de usar e de estruturas muito poderosas e ricas no próximo capítulo.

Falarei sobre e levarei todos vocês em um breve histórico de desenvolvimento de aplicativos para dispositivos móveis, como evoluíram ao longo dos tempos e por que você, como um possível desenvolvedor da ABB, pode se tornar conhecido nesse campo por meio do Google furter ou talvez de algum outro SDK você está confortável com o uso.

Então vamos seguir em frente.


Seção 2: A História do Desenvolvimento de Aplicativos Móveis

4. Conhecendo seu histórico de aplicativos

Como mencionei, o desenvolvimento de aplicativos cresceu muito nos últimos anos.

Como alguém que está interessado em entrar nesse campo, será bom ter uma visão geral de como o mercado de desenvolvimento de aplicativos para dispositivos móveis cresceu.

Quando eles foram introduzidos pela primeira vez, os telefones celulares foram completados como tecnologia que era usada apenas para fazer ligações telefônicas.

No entanto, mais tarde, como sabemos que o jogo mudar a invenção de smartphones levou à abertura da porta de desenvolvimento de aplicativos móveis, os aplicativos de software são como os chamamos de trabalho para levá-los a executar ou operar em dispositivos móveis modernos, como tablets e smartphones.

Com o passar dos anos, esses aplicativos foram aprimorados e agora parece ter se tornado uma parte importante de nossas vidas.

Eles parecem ter se integrado perfeitamente ao nosso estilo de vida, começando pelo início do celular.

Vamos começar isso de todo o caminho de volta.

Sim, estou falando do começo do celular e sim da primeira ligação de celular feita.

Para aqueles de vocês que não conhecem a primeira ligação de celular já feita, em 3 de abril de 1973, a mesma ligação foi feita por Martin Cooper, da Motorola.

De fato, esse telefonema era basicamente um golpe de publicidade para a grande empresa.

Não foi até 10 anos depois, desde que a bola que o primeiro telefone celular atingiu oficialmente o mercado, mesmo assim, o jornal público não foi capaz de usá-lo.

Por quê.

Bem, porque a morte não é o primeiro telefone celular no mercado custou uma grana por \$ 2000 e pesava aproximadamente £ 2.

Sim.

Isso é muito.

Agora, é claro, esse modelo inicial da Foon não tinha nenhum aplicativo.

Quero dizer, as pessoas provavelmente nem sabiam sobre apps naquela época, foi na década de 1990, quando o BBH foi introduzido como os primeiros sistemas operacionais que permitiam aplicativos durante esse tempo específico.

Esses dispositivos eram aqueles que incluíam processadores de texto e bancos de dados de planilhas do diário.

É claro que a tecnologia melhorou e o TAAS BDA é um trabalho para se tornar acessível ao lidar com mais aplicativos.

Era uma linguagem de programação aberta que possibilitava aos usuários criar seus próprios aplicativos para o desenvolvimento de aplicativos personalizados.

As empresas precisam de dispositivos Foster e sim de sistemas operacionais ainda melhores.

Você pode não perceber isso agora, mas o lançamento do smartphone BlackBerry, que não chegou lá em 2002, foi anunciado como uma grande conquista para as empresas de tecnologia.

Esse tipo de telefone foi capaz de integrar perfeitamente o email sem fio e outros recursos que foram os primeiros de muitos durante esse período.

Java m e era muito popular para telefones e Beedi é por isso que você pode perguntar.

Isso porque permitia aos usuários espaço adicional na memória.

Então, em 2009, o lançamento do Symbian abriu as portas para novos desenvolvimentos ainda mais de acordo com os dados, pelo menos, 50 milhões de dispositivos adotaram o sistema operacional Symbian depois que foi lançado.

Mesmo aparelhos da Nokia, aparelhos da Samsung e até mesmo telefones LG começaram a usar o novo sistema operacional para melhorar a si mesmos como marcas.

Então, chegando ao desenvolvimento de aplicativos personalizados que atingiram o mainstream Bem, uma vez que os desenvolvedores começaram o desenvolvimento começou a acelerar.

Não demorou muito para que os apps atingissem o mainstream e o tempo do Grant.

Vivemos no.

Nós basicamente temos um aplicativo para cada tanque.

No entanto, voltando à história de tudo em 2007, o primeiro iPhone foi lançado pela Apple, a App Store foi adicionado pela empresa muito em seguida, permitindo aos usuários encontrar baixar e usar aplicativos.

No começo, esses aplicativos eram limitados, mas não demorou muito para que os desenvolvedores de aplicativos descobrissem o potencial inexplorado nesse campo.

Seguindo o exemplo, o mercado Android deu às pessoas uma outra plataforma para acessar aplicativos e até hoje há uma competição entre o aplicativo da Apple e do Android e eu acho que é para continuar.

E enquanto a competição entre o Android e a Apple continua e os desenvolvedores olham para as plataformas planas da base de clientes potenciais, forneça-as.


Seção 3: Por que usar o Flutter?
5. Entendendo por que você deve considerar o uso do Google
Flutter


Duvido que você conheça um pouco mais sobre o slurper e como ele pode ajudar você como desenvolvedor de aplicativos.

Vamos ver porque você deveria usar o Slichter.

Como mencionado antes, o flirter é o K Isso significa um kit de desenvolvimento de software.

É considerado por muitos como uma maneira eficiente de criar aplicativos móveis de plataforma cruzada com uma interface de usuário ou interface do usuário impressionante.

Observe que a maneira desordenada de criar visualizações tem semelhanças com o aplicativo da web.

É aí que você pode encontrar várias analogias para DML e CFS.

De acordo com os desenvolvedores do selector schlechter torna mais fácil e rápido para construir um belo aplicativo móvel D-Wave flicker para a maioria das células.

Parece bom.

No entanto, para aqueles de vocês que talvez não saibam sobre outras soluções de plataforma cruzada, como Zahm nativo de re-ação iônica, você pode querer saber um pouco mais sobre como o Slichter pode ajudar.

Existem certos benefícios ou o flerte Stec pode ajudá-lo a mudar de ideia.

E você sabe que você precisa começar a usá-lo.

Alguns de vocês podem ter muitas perguntas sobre flertes agora.

Como funciona a carta?

Por que é considerado inovador pelos outros?

Por que eles deveriam usá-lo?

Etc ..

E espero responder a todos eles no discurso.

E embarque no benefício de o flicker ser o SDK imóvel para a criação de aplicativos móveis de plataforma cruzada Krok é que você pode escrever um código e depois executar o aplicativo no iOS e no Android.

O código que você escreve tem que ser escrito no escuro.

Então, qualquer idioma desenvolvido pelo Google Docs parecerá familiar para você.

Se você já teve experiência de usar Java antes.

Tome nota que, em vez de usar arquivos SML, você constrói um layout.

Tal lay out é construído a partir de componentes de widgets que são aninhados seu widget é um aplicativo de material que é o aplicativo inteiro e, em seguida, você tem desconforto que é Dumaine lay out estrutura, em seguida, movendo-se dentro.

Você tem a barra de aplicativos que é como a barra de ferramentas do Android 2 e, em seguida, um contêiner como o corpo.

Agora, é nesse corpo que você coloca os widgets layouts, como botões, texto, etc., avançando ou por que você deve considerar o uso do clicker. Veja uma lista.

Número um de recarga quente.

Como mencionado antes, o recarregamento do Hawks pode ser considerado como outro melhor recurso que é usado para recarregar.

Você pode criar instantaneamente os projetos em que está trabalhando, como se estivesse construindo uma página da Web, simplesmente alterando algo no código.

Clique em carga de Hawtry e você poderá ver o resultado instantaneamente.

Número dois um conjunto inteiro de widgets de design de material que flertam você começa a obter a ecosfera registrada com construído em componentes de interface do usuário, você deve tomar nota de que existem dois conjuntos de rejeições.

Há o design do material que é para o Android e o Cupertino do Derrida, que é para iOS.

Você simplesmente tem que selecionar o widget que você quer e depois ir a partir daí.

Também não há necessidade de você se preocupar com o fato de os widgets serem muito específicos.

Isso significa que, mesmo que você acabe implementando alguns Cupertino, os widgets projetados por Maduna vão acabar parecendo iguais em todos os dispositivos iOS e Android existentes.

Portanto, não há necessidade de você se preocupar com algo em seu aplicativo ou até mesmo com o aplicativo como um todo, parecendo diferente.

Quando executado em dispositivos diferentes No entanto, você pode usar diferentes equipes Android e iOS, se desejar.

Também tudo é um widget em flertar.

Sim, até mesmo a sua classe de aplicativo, o aplicativo de material é um widget.

Então, toda a sua estrutura de layouts é escaneada por dimensão.

Então, ter tudo como um widget permite que você use flirter para criar UI impressionante de uma maneira muito simples.

Numero tres.

Vários pacotes, apesar de terem sido lançados mortos.

Recentemente, em 2007, a comunidade deflectiva de Dean é muito ativa e envolvida em torná-la melhor.

Isso fez com que o flatter fosse capaz de suportar muitos pacotes que você tem acesso a pacotes para abrir imagens compartilhando a corrente e fazer solicitações de HDTV acessando sensores armazenando suas preferências.

Implementando Firebase e a lista continua.

Todo esse suporte é para iOS e Android.

Agora que você passou por cima dos benefícios do sluttier, pergunte a si mesmo. Você está interessado em usar o flirter?

Não há necessidade de tomar uma decisão precipitada.

Estarei falando mais sobre tudo isso enquanto o discurso continua.


Seção 4: Como instalar o Flutter?
6. Instalando o Flutter para Windows, Linux e MacOS

Número um dos seus sistemas operacionais deve ser o Windows 7 S-B 1 são mais tarde de 64 bits, você precisa de pelo menos 400 M-B ou três espaço em disco.

Tome nota que não inclui este lugar para Dools.

Falando sobre as ferramentas que você precisa.

Ponto de shellfire Bovver.

Ou você não precisa.

Este é o G ID para o Windows.

Quando eles usam boa opção de prompt de comando do Windows.

Se você tiver para o Windows já está instalado.

Precisa ter certeza de que você pode executar o comando get para o prompt de comando ou instalar o flirter no Mac

SO para instalar e executar o flirter no Mac OS, seu ambiente de desenvolvimento deve atender a esses requisitos mínimos.

Estes são o sistema operacional deve ser, pelo menos, Mac OS 64 bits, pelo menos, 700 M-B de espaço livre em disco.

Novamente, isso não inclui espaço em disco para Dools.

Agora, observe que o flertador depende de várias ferramentas de linha de comando disponíveis em seu ambiente.

Estes são Bash e Diyar r m.

Boa menina.

Descompacte e até mesmo a árvore numérica.

Continuando a instalar o flirter no Linux para instalar e executar o flirter no Linux.

Você precisa de um sistema operacional que seja o Linux 64 bit 600 M-B de espaço livre em disco.

Isso não inclui espaço em disco para o seletor Dools, dependendo de algumas ferramentas de linha de comando disponíveis

em seu ambiente.

Estes são novamente Bash e ganham Diyar.

R m.

Boa menina.

Descompacte e qual comando de teste flexor depende da biblioteca estar disponível em seu ambiente que

é como você pode ver lib G.O. você Daut s o dot um fornecido por mim de bagagem, por exemplo l AB lib Glou one me no Ubuntu são Debian.

Você pode vê-lo no slide, então não deixe de reservar o site oficial do seletor.

Você tem uma ideia mais aprofundada sobre o procedimento que precisa ser seguido para instalar e começar a usar o flexor.



Seção 5: Benefícios do Flutter
7. Benefícios do Flutter

Com base em uma parte do Curso onde eu falei sobre o motivo pelo qual você deve considerar o uso do flirter um pouco mais sobre todos os benefícios e pode oferecer a você muitos desenvolvedores recomendam usar o flirter.

Se você está apenas começando no campo de desenvolvimento de aplicativos por causa de sua natureza amigável quando se trata de criar aplicativos multiplataforma, vamos passar por cima de uma lista.

Flirter de projeto de código aberto número um é um projeto de código aberto que permite que ele esteja disponível para uso também por startups.

Ele oferece muitas opções para criar o tipo de aplicativo que você deseja criar para Android ou iOS.

Além disso, sendo de código aberto, dá origem à colaboração aberta, permitindo que outros ajudem a melhorar ou faça o número dois de Ed impressionantes. integração usando flirter.

Você pode continuar e continuar adicionando e subtraindo edições durante o processo de desenvolvimento do aplicativo.

Oferece uma impressionante integração de edição com o estúdio Android e o código do Visual Studio.

Isso leva a conclusões mais inteligentes baseadas em onde os tipos de definições usuais e o número de módulos importados criam campos escuros como o Java.

Por que a dieta dietética não é uma cópia direta do Java.

Os dois são semelhantes a similaridades.

Ajude os desenvolvedores a mudar para o escuro.

Se eles estão usando Java para o número quatro da lista, temos 4 gerentes de engenharia que gerentes de engenharia também podem usar o flirter se tiverem a necessidade de liderar mais equipes de desenvolvimento biológico usando um SDK como gerente de engenharia.

Você pode criar uma equipe surda de um único aplicativo móvel ou uma equipe de desenvolvimento para ajudar você a unificar os investimentos em desenvolvimento.

Você pode enviar recursos mais rapidamente para reduzir seus custos de manutenção e transferir os mesmos recursos para o Android e o iOS simultaneamente.

Em suma, quando você está usando a desordem, você tem acesso a melhores recursos de widgets fáceis de usar e integrações editoriais impressionantes para o desenvolvimento de aplicativos.

Não é de admirar que muitos desenvolvedores tenham usado o seletor para criar aplicativos móveis, mas também para alcançar um público mais amplo.

No Android e no iOS.

Seção 6: Quanta Experiência como Desenvolvedor de Aplicativos
Devo ter que usar o Flutter?

8. Quanta experiência como desenvolvedor de aplicativos eu deveria
tem que usar o Flutter?

Esta é a pergunta que muitos de vocês podem ter atualmente.

Em suma, você não precisa de nenhuma experiência anterior para aprender e usar flirter.

Este SDK é muito acessível para programadores que têm uma ideia sobre conceitos de programação imperativa e conceitos orientados a objetos.

Contanto que você tenha paixão e interesse em usar a desordem para criar aplicativos, você achará o processo bastante fácil e rápido.

Vamos falar sobre isso um pouco mais.

Número um do estilo reativo ao usar o flirter, você precisará começar em algum lugar para criar um aplicativo para iOS e Android.

É por isso que você precisa estar pronto para trabalhar usando uma estrutura reativa.

Número fazer.

David você não pode falar sobre seletor sem falar sobre a arquitetura de widgets que oferece.

Como mencionado tudo é o widget e seletor de seu elemento de estrutura para eles e você é o botão dois elemento muito estilístico que você deseja usar, como um esquema de cores e formatar tudo.

É um isolante rígido que permite criar o tipo de aplicativo que você quer sem problemas.

Você ganha muito controle sobre a renderização do objeto de composição de execução do aplicativo e mais número três.

Diga adeus ao gerenciamento do ciclo de vida da atividade.

Se você é alguém que se envolveu um pouco no desenvolvimento de aplicativos, você estará familiarizado com o gerenciamento do ciclo de vida da atividade.

Ele só aparece como trabalho extra, não é?

Uma atividade periférica, se possível, que não seja adicionada diretamente no processo geral de produção de seu aplicativo.

No entanto, o que flertar você não terá que se preocupar com o gerenciamento do ciclo de vida da atividade.

Isso ocorre porque o patch do conector de desordem ajuda a fazer com que todas as atividades transmitidas sejam vestidas e carregadas de maneira síncrona.

Número de 60 SPSS consistentes são quadros por segundo.

O Flirter também é útil para quem procura criar um seletor de aplicativos gráficos ou de jogos pesados, capaz de oferecer uma taxa consistente de 60 quadros por segundo devido à natureza reativa e ao uso de suporte sombrio.

Você terá visuais estáveis ??em seu aplicativo desenvolvido.

O número cinco, o flertador de suporte da comunidade, também tem uma grande comunidade de expedientes, além de demitir desenvolvedores que podem ajudá-lo com o suporte necessário que você precisa.

A comunidade torna mais fácil para você aprender usando o flerte, bem como fazer conexões úteis no campo de desenvolvimento de aplicativos.

Então, vamos falar um pouco sobre darte e por que Slichter usa o começo.

Seção 7: Conhecendo o Dart
9. O que é o Dart e por que usá-lo?

Durante este curso eu mencionei d'arte muito como uma pequena atualização.

Dart é uma linguagem de programação de uso geral que foi originalmente desenvolvida pelo Google.

Mais tarde foi aprovado como padrão pela ECMA daks ECMA para 8 para ser mais preciso.

Você pode usar d'arte para construir aplicações web e móveis para servidores.

Então, por que os desenvolvedores do d'arte trabalham no Google, além de outros.

Não use darte para criar aplicativos de alta qualidade para iOS.

E Android, assim como a web.

Então é por isso que pode ser visto ou é visto como uma ótima opção quando se trabalha com recursos voltados para o desenvolvimento do lado do cliente.

Vamos a uma lista sobre o Bart.

A sintaxe escura produtiva número um é clara e concisa.

O ferramental geral é muito simples, mas poderoso.

Você pode identificar erros sutis precocemente através da digitação de sons usando darte.

Você também pode obter inúmeras bibliotecas de pontuação e milhares de pacotes.

Número dois rápido Dart pode ser denominado como uma linguagem que oferece otimização antes da violação.

Isso ajuda você a obter desempenho previsivelmente alto e inicialização rápida em dispositivos móveis e na Web.

O número três viável para aplicativos móveis escuros é executado de forma nativa no Android e no iOS.

Compiladores sombrios fazem A-R como um código X 86.

Tome nota DEC para web app caiu spiles para javascript e é acessível.

Muitos desenvolvedores estão familiarizados com o d'arte que você faz.

Não é uma orientação e sintaxe de objeto sem surpresas.

Você pode facilmente pegar o jeito escuro se você já sabe sobre Java são C ++.

Agora, se você sabe um pouco mais sobre d'arte Lexi, por que você deve considerar aprendê-lo em 2018, bem como por que o flerte faz uso do efeito.

Seção 8: Por que você deve considerar o aprendizado de dardo em
2018

10. Se você considerar a aprendizagem do dardo em 2018 e
Além?

Se você tem pensado em aprender, aqui está uma lista que pode ajudá-lo a tomar uma decisão.

Número um fácil de aprender.

Como mencionado anteriormente, os guardas de Daryn compartilham semelhanças com outras linguagens comuns.

Isso também significa que você já deve saber como usar d'arte sem perceber que será fácil para você aprender se tiver experiência com outras linguagens orientadas a objetos, como Java, são C ++.

Mesmo se você souber alguma aprendizagem sobre javascript, Darb não será difícil para você.

Outra grande coisa é que o dardo suporta tanto a digitação solta quanto a forte.

Isso permite que seja mais fácil quando você está mudando de outro idioma.

Number faz uma base de código compartilhada nativamente compilada.

Alguns de vocês podem saber que outras estruturas permitem que você compartilhe partes da base de código em diferentes plataformas.

No entanto escuro é um pouco diferente.

Ele permite que você escreva um único aplicativo que pode ser usado tanto no iOS quanto no Android, juntamente com a maneira como ele é compilado nativamente pelo Dubost.

Agora, é claro, isso é algo que os outros podem fazer, mas o darte é capaz de permitir ainda mais o compartilhamento.

Você sabe que eu posso dizer que permite que você compartilhe um pouco do seu código entre a inteireza do seu GUI, não apenas os aplicativos móveis fazem o Menke para diferentes plataformas, significando iOS e Android, mas também para a Web, bem como outras aplicações de guarda também .

Isso permite que você compartilhe a maior parte do código não UI ou digamos que o código específico em preto com o número de frameworks do Underdark seja produtivo. Usando o escuro, você terá a experiência de como é fácil e rápido criar o layout e adicionar funcionalidade para o seu aplicativo.

Você pode criar o Caly no código criando uma série de widgets.

Lembre-se do que eu disse a você que, em flertar, tudo o que o widget usa com essas ferramentas certamente ajudará você a criar o que quiser no menor tempo possível.

Número quatro compila antes do tempo e apenas no tempo quando você está desenvolvendo um aplicativo Avodart você pode ver as mudanças que você faz instantaneamente.

Isso significa que você não terá que passar pelo procedimento de recompilação e não há necessidade de esperar que o aplicativo seja recarregado.

Você só precisa salvar a alteração e ver instantaneamente as alterações.

Isso é possível porque, com essa estrutura de desenvolvimento de aplicativo para celular, você pode compilar antecipadamente.

R E ou D, bem como na hora certa.

R d.

Agora que você está ciente das razões mencionadas, você provavelmente está começando a entender o benefício de usar o darte.

Experimente e veja se é algo sobre o qual você deseja aprender mais.

Como um desenvolvedor de aplicativos que está mudando meu discurso, eu deixo você saber por que flirter escolheu usar d'arte.

Seção 9: Por que o Flutter usa Dart?

11. Por que o Flutter Decidiu Usar o Dart?
Flirter decidir usar d'arte não foi uma decisão precipitada.

Flirter usa essa ferramenta.

Um total de quatro dimensões primárias para avaliação e considerou todas as necessidades dos autores, desenvolvedores e usuários finais do framework.

Foi fundada enquanto algumas línguas cumpriam certos requisitos.

Estava escuro lá.

Marcou muito em todas as dimensões de avaliação que foram verificadas pela equipe mais plana e também atendeu a todos os requisitos e critérios determinados.

Você deve saber que os dubladores de Darkthrone e compiladores suportam a combinação de dois recursos críticos para flertar um parding ao tema de flutter.

O primeiro é o ciclo de desenvolvimento rápido baseado em GAAP, que permite a mudança de forma e a carga de Hawtry com estado na linguagem com tipos.

Como eu mencionei anteriormente, o GAAP está na hora certa.

A segunda é a antecipação de nosso descompilador imenso, que emite uma importação A-R eficiente para uma inicialização rápida e desempenho previsível de implementações de produção.

Além disso, a equipe de flirter foi capaz de trabalhar de perto com a comunidade DAR.

A comunidade é ativa e continua a investir recursos para melhorar o desenho quando se trata de usá-lo como exemplo.

Quando a equipe adotada darte decidiu definhar não teve uma hora antes do tempo ou toolbol AOB Freberg usando binário nativo é um tal recurso é fundamental para alcançar alto desempenho previsível.

No entanto, agora a linguagem tem isso porque eles são dark deam e eles estão construindo para Lucker no dia primário um grande dia com dark marcou uma pontuação alta para a vibração deam foram número um produtividade do desenvolvedor de acordo com o tema.

Uma das principais proposições de valor de letras é que ele economiza recursos de engenharia, permitindo que desenvolvedores criem aplicativos para iOS e Android com a mesma base de código usando uma linguagem altamente produtiva que acelera os desenvolvedores Forder e torna a carta mais atraente para eles.

Isso foi muito importante para a equipe de framework defletor da Dubow, bem como para os desenvolvedores.

Você deve saber que a maior parte da sorte é construída na mesma linguagem dada aos usuários.

Portanto, o tema precisa permanecer produtivo em cem mil linhas de código sem sacrificar a acessibilidade ou a legibilidade da estrutura e dos widgets para os desenvolvedores.

Orientação número dois objeto para flirter que vorn tema fez uma linguagem que foi adequado para flusters domínio do problema que é bom outra coisa que você vai usar suas experiências de acordo com a equipe da indústria tem várias décadas de conveniência construindo frameworks de interface do usuário em linguagens orientadas a objeto.

Embora eles possam ter usado uma linguagem conhecida orientada a objetos, isso significaria reinventar a roda para resolver vários problemas difíceis.

Além disso, a grande maioria dos desenvolvedores já tem experiência com desenvolvimento orientado a objetos, tornando mais fácil aprender como desenvolver seus aplicativos, cujo seletor agora possui um alto desempenho previsível de acordo com o refletor de temas.

Eles querem capacitar os desenvolvedores para criar experiências de usuário rápidas e fluidas.

Então, para que eles cumpram essa promessa ou objetivo, eles precisam ser capazes de executar uma quantidade significativa de placa de desenvolvedor durante cada quadro de animação.

Isso significa que a equipe precisava de uma linguagem que pudesse fornecer alto desempenho e fornecer desempenho previsível sem bater seus chefes.

Isso poderia causar o número de quadros perdidos para localização fasc neste ponto.

Você deve saber que a estrutura do defletor utiliza um fluxo de estilo funcional que, de acordo com o tema da letra, depende muito do eficiente alocador de memória subjacente que manipula com eficiência o pequeno local de curta duração.

Este arquivo foi desenvolvido em linguagens com essa propriedade e não funciona de maneira eficiente em idiomas como esse recurso.

Então agora você sabe por que flutter decidiu usar d'arte e por que você deveria estar familiarizado com tal linguagem.

E quem sabe você pode querer usá-lo também.


Seção 10: Perguntas Freqüentes sobre o Flutter
12. O que há dentro do Flutter SDK?
Permita-me ir com freqüência para fazer perguntas que você possa ter em mente que desapontam.

A primeira pergunta é o que está dentro do SDK flertador.

Quando se trata de responder o que está dentro do deflector SDK.

Aqui está a lista.

O número um otimizou fortemente o primeiro mecanismo de renderização de leads duplo da Mumbai com excelente suporte para texto.

Em seguida, temos o conjunto de widgets de estilo de re-ato de Morgan o conjunto rico de widgets para Android e iOS que eu disse anteriormente tudo é um widget em flertar.

Número quatro, é bom para testes de unidade e integração.

Em seguida, entramos em interoperabilidade e plug-in API para conectar-se ao sistema e participar D DST gais headless best runner para executar testes em ferramentas de linha de comando do Windows Linux e Mac para criar testes de construção e compilar seus aplicativos.

13. O Flutter vem com uma estrutura?

Faz flutter vem com o quadro.

Sim.

Como desenvolvedores de aplicativos, você deve saber que os sistemas de terra plana com uma estrutura moderna inspirada pela estrutura de aletas de re-ação foram projetados para serem lidos e personalizáveis ??como um desenvolvedor.

Você pode ir em frente e optar por usar apenas parte do framework que você tem acesso, você é um framework completamente diferente.

A escolha é sua.


14. O Flutter vem com widgets?
Faz flutter vem com Vineyard como mencionado antes que sacode no parque e parte de flutter.

Então, sim, selecione seus navios com um conjunto de alta qualidade McKeel design em layouts e equipes de widgets Cupertino.

Você também pode ir em frente e fazer seus próprios widgets ou até mesmo personalizar qualquer um dos widgets existentes para criar o tipo de aplicativo que você deseja.

Novamente lembre-se tudo é um widget e flutter.


15. Como o Flutter roda meu código no Android?

Como flertar meu no Android.

Alguns de vocês podem estar se perguntando como o Slichter executa seu código e o direito de lhe dar uma resposta simples.

De acordo com a equipe do flooder, o código Engine C e C ++ são compilados com andróides e DK The Dark Lord.

Boada SDK do que o seu está à frente do tempo.

AOP compilado em uma biblioteca ARMM nativa.

Essa biblioteca está incluída em um projeto de corredor e dreich e a coisa toda é construída em qualquer B.K. quando é lançado, o





\nota{Aqui preciso colocar os trabalhos que j� encontrei sobre an�lise de tetraz�lio por imagem}

Ao constatar a import�ncia que o teste de tetraz�lio tem no processo de qualidade das sementes, e o impacto que gera na produ��o de soja no Brasil, � importante pensarmos em como a computa��o pode auxiliar no processo de analise de sementes no teste de tetraz�lio.

.

.



Tetrazolio - Soja e outros \nota{Aqui que apresentar alguns trabalhos com tretaz�lio em soja e tamb�m em outras sementes, mostrado que outras culturas podem ser exploradas futuramente.}

.


.



Alguns trabalhos foram propostos na ultima d�cada, com a inten��o de usar vis�o computacional no auxilio da identifica��o dos danos de soja no teste de tetraz�lio \nota{quero apresentar aqui os trabalhos que desenvolveram algum software espec�fico para an�lise de tetraz�lio e os m�todos utilizados}


como \citeauthor{Rocha2016} usando o m�todo de ....;\\
BBB usando o m�todo de ....\\
CCC usando o m�todo ...
	
.


.




Assim como \cite{Wendt2014, Junior2017} ... \nota{Apresentar trabalhos que usaram softwares gen�ricos de an�lise de imagem }


.

.



{\small \textst{escrever sobre base de dados para pesquisa }} \\

.




%processamento de imagens 

%deep learn

%{\small \textst{Uso de software n�o especifico}} \\

\section{a}

Assim como \cite{Wendt2014, Junior2017}



%

\section{Flutter}


%%%%%%%%%%%%%%%%%%%

%%%%%%%%%%%%%%%%%%%

Eu vou estar indo sobre o básico de Dart, especialmente porque é usado por flutter e mais desenvolvimento de aplicativos ganhou muita tração ao longo dos anos e há um monte de lucro a ser ganho.


Eu começo falando sobre flutter e depois me movo para o dardo.



2. O que é Flutter?

Vamos começar com o primeiro take o que é flirter para aqueles de vocês que talvez não conheçam o Google Flirter como o nome indica é SDK de aplicativo móvel do Google, que pode ser usado para criar interfaces nativas de alta qualidade em iOS e Android em um curto período de tempo.

O lançamento inicial do flirter foi em maio de 2017.

Também uma nota lateral SDK significa desenvolvimento de software.

Boa.

Você deve saber que o flirter faz uso do código existente para funcionar.

Os muitos benefícios oferecidos o levaram a ser usado por desenvolvedores de aplicativos independentes e até por organizações famosas em todo o mundo.

Também é livre para usar uma fonte aberta.

Algumas das coisas básicas sobre o flerte que você deve tomar nota.

Nosso número um desenvolvimento rápido usando flirter.

Você pode ter recarregado em uma quantidade de milissegundos para garantir que seu aplicativo possa ganhar vida no menor tempo possível.

Você também pode usar uma vasta gama de widgets totalmente personalizáveis ??que ajudam a criar interfaces nativas da maneira mais rápida possível.

Eu estou falando de ser capaz de construir em minutos aqui.

Número dois interface expressiva e flexível são interface do usuário.

Seu aplicativo não pode ser bem-sucedido se não for fácil de usar para as pessoas.

Os melhores aplicativos lá fora também apresentavam a melhor experiência e a experiência do usuário.

Então, o que você conseguir mais ferramentas para garantir que o nativo e experiência do usuário é o melhor que pode ser.

A arquitetura em camadas oferece acesso à personalização completa, permitindo que você crie aplicativos interativos e amigáveis ??ao usuário, que têm uma renderização de renderização incrivelmente rápida, além de um expressivo desempenho nativo da árvore de números.

Independentemente de uma pessoa estar usando um dispositivo Android e iOS, seu aplicativo precisa oferecer a experiência que eles esperam que os widgets no flertador possam incorporar todas as diferenças críticas de plataforma para garantir a qualidade.

Essas diferenças negras incluem ícones de rolagem de navegação e garfos para a incorporação do DS às diferenças. O desempenho nativo de seus aplicativos em dispositivos Android e iOS é mantido no nível ideal.

Agora que você adquiriu um conhecimento básico do Google Flirter, deixe-me dar uma rápida introdução indireta ao escuro.



3 O que é o Dart?

O que é obscuro quando se trata de entender darte você deve saber que já é um objeto e fez e florescer a linguagem definida que você usa a sintaxe no estilo C que compila opcionalmente no javascript d'arte oferece muito apoio.

Isso significa que ele pode suportar interfaces de classes abstratas que fazem sentido digitação estática e até mesmo um sistema de tipo de som.

Agora, todos esses termos podem soar um pouco demais para aqueles que não estão familiarizados com esse jargão.

Bem, deixe-me tornar as coisas mais fáceis para você.

Basicamente escuro ajuda você a criar lindas experiências de alta qualidade em todas as telas de dispositivos por meio de uma linguagem que nega a otimização de ferramentas flexíveis fáceis de usar e de estruturas muito poderosas e ricas no próximo capítulo.

Falarei sobre e levarei todos vocês em um breve histórico de desenvolvimento de aplicativos para dispositivos móveis, como evoluíram ao longo dos tempos e por que você, como um possível desenvolvedor da ABB, pode se tornar conhecido nesse campo por meio do Google furter ou talvez de algum outro SDK você está confortável com o uso.

Então vamos seguir em frente.


Seção 2: A História do Desenvolvimento de Aplicativos Móveis

4. Conhecendo seu histórico de aplicativos

Como mencionei, o desenvolvimento de aplicativos cresceu muito nos últimos anos.

Como alguém que está interessado em entrar nesse campo, será bom ter uma visão geral de como o mercado de desenvolvimento de aplicativos para dispositivos móveis cresceu.

Quando eles foram introduzidos pela primeira vez, os telefones celulares foram completados como tecnologia que era usada apenas para fazer ligações telefônicas.

No entanto, mais tarde, como sabemos que o jogo mudar a invenção de smartphones levou à abertura da porta de desenvolvimento de aplicativos móveis, os aplicativos de software são como os chamamos de trabalho para levá-los a executar ou operar em dispositivos móveis modernos, como tablets e smartphones.

Com o passar dos anos, esses aplicativos foram aprimorados e agora parece ter se tornado uma parte importante de nossas vidas.

Eles parecem ter se integrado perfeitamente ao nosso estilo de vida, começando pelo início do celular.

Vamos começar isso de todo o caminho de volta.

Sim, estou falando do começo do celular e sim da primeira ligação de celular feita.

Para aqueles de vocês que não conhecem a primeira ligação de celular já feita, em 3 de abril de 1973, a mesma ligação foi feita por Martin Cooper, da Motorola.

De fato, esse telefonema era basicamente um golpe de publicidade para a grande empresa.

Não foi até 10 anos depois, desde que a bola que o primeiro telefone celular atingiu oficialmente o mercado, mesmo assim, o jornal público não foi capaz de usá-lo.

Por quê.

Bem, porque a morte não é o primeiro telefone celular no mercado custou uma grana por \$ 2000 e pesava aproximadamente £ 2.

Sim.

Isso é muito.

Agora, é claro, esse modelo inicial da Foon não tinha nenhum aplicativo.

Quero dizer, as pessoas provavelmente nem sabiam sobre apps naquela época, foi na década de 1990, quando o BBH foi introduzido como os primeiros sistemas operacionais que permitiam aplicativos durante esse tempo específico.

Esses dispositivos eram aqueles que incluíam processadores de texto e bancos de dados de planilhas do diário.

É claro que a tecnologia melhorou e o TAAS BDA é um trabalho para se tornar acessível ao lidar com mais aplicativos.

Era uma linguagem de programação aberta que possibilitava aos usuários criar seus próprios aplicativos para o desenvolvimento de aplicativos personalizados.

As empresas precisam de dispositivos Foster e sim de sistemas operacionais ainda melhores.

Você pode não perceber isso agora, mas o lançamento do smartphone BlackBerry, que não chegou lá em 2002, foi anunciado como uma grande conquista para as empresas de tecnologia.

Esse tipo de telefone foi capaz de integrar perfeitamente o email sem fio e outros recursos que foram os primeiros de muitos durante esse período.

Java m e era muito popular para telefones e Beedi é por isso que você pode perguntar.

Isso porque permitia aos usuários espaço adicional na memória.

Então, em 2009, o lançamento do Symbian abriu as portas para novos desenvolvimentos ainda mais de acordo com os dados, pelo menos, 50 milhões de dispositivos adotaram o sistema operacional Symbian depois que foi lançado.

Mesmo aparelhos da Nokia, aparelhos da Samsung e até mesmo telefones LG começaram a usar o novo sistema operacional para melhorar a si mesmos como marcas.

Então, chegando ao desenvolvimento de aplicativos personalizados que atingiram o mainstream Bem, uma vez que os desenvolvedores começaram o desenvolvimento começou a acelerar.

Não demorou muito para que os apps atingissem o mainstream e o tempo do Grant.

Vivemos no.

Nós basicamente temos um aplicativo para cada tanque.

No entanto, voltando à história de tudo em 2007, o primeiro iPhone foi lançado pela Apple, a App Store foi adicionado pela empresa muito em seguida, permitindo aos usuários encontrar baixar e usar aplicativos.

No começo, esses aplicativos eram limitados, mas não demorou muito para que os desenvolvedores de aplicativos descobrissem o potencial inexplorado nesse campo.

Seguindo o exemplo, o mercado Android deu às pessoas uma outra plataforma para acessar aplicativos e até hoje há uma competição entre o aplicativo da Apple e do Android e eu acho que é para continuar.

E enquanto a competição entre o Android e a Apple continua e os desenvolvedores olham para as plataformas planas da base de clientes potenciais, forneça-as.


Seção 3: Por que usar o Flutter?
5. Entendendo por que você deve considerar o uso do Google
Flutter


Duvido que você conheça um pouco mais sobre o slurper e como ele pode ajudar você como desenvolvedor de aplicativos.

Vamos ver porque você deveria usar o Slichter.

Como mencionado antes, o flirter é o K Isso significa um kit de desenvolvimento de software.

É considerado por muitos como uma maneira eficiente de criar aplicativos móveis de plataforma cruzada com uma interface de usuário ou interface do usuário impressionante.

Observe que a maneira desordenada de criar visualizações tem semelhanças com o aplicativo da web.

É aí que você pode encontrar várias analogias para DML e CFS.

De acordo com os desenvolvedores do selector schlechter torna mais fácil e rápido para construir um belo aplicativo móvel D-Wave flicker para a maioria das células.

Parece bom.

No entanto, para aqueles de vocês que talvez não saibam sobre outras soluções de plataforma cruzada, como Zahm nativo de re-ação iônica, você pode querer saber um pouco mais sobre como o Slichter pode ajudar.

Existem certos benefícios ou o flerte Stec pode ajudá-lo a mudar de ideia.

E você sabe que você precisa começar a usá-lo.

Alguns de vocês podem ter muitas perguntas sobre flertes agora.

Como funciona a carta?

Por que é considerado inovador pelos outros?

Por que eles deveriam usá-lo?

Etc ..

E espero responder a todos eles no discurso.

E embarque no benefício de o flicker ser o SDK imóvel para a criação de aplicativos móveis de plataforma cruzada Krok é que você pode escrever um código e depois executar o aplicativo no iOS e no Android.

O código que você escreve tem que ser escrito no escuro.

Então, qualquer idioma desenvolvido pelo Google Docs parecerá familiar para você.

Se você já teve experiência de usar Java antes.

Tome nota que, em vez de usar arquivos SML, você constrói um layout.

Tal lay out é construído a partir de componentes de widgets que são aninhados seu widget é um aplicativo de material que é o aplicativo inteiro e, em seguida, você tem desconforto que é Dumaine lay out estrutura, em seguida, movendo-se dentro.

Você tem a barra de aplicativos que é como a barra de ferramentas do Android 2 e, em seguida, um contêiner como o corpo.

Agora, é nesse corpo que você coloca os widgets layouts, como botões, texto, etc., avançando ou por que você deve considerar o uso do clicker. Veja uma lista.

Número um de recarga quente.

Como mencionado antes, o recarregamento do Hawks pode ser considerado como outro melhor recurso que é usado para recarregar.

Você pode criar instantaneamente os projetos em que está trabalhando, como se estivesse construindo uma página da Web, simplesmente alterando algo no código.

Clique em carga de Hawtry e você poderá ver o resultado instantaneamente.

Número dois um conjunto inteiro de widgets de design de material que flertam você começa a obter a ecosfera registrada com construído em componentes de interface do usuário, você deve tomar nota de que existem dois conjuntos de rejeições.

Há o design do material que é para o Android e o Cupertino do Derrida, que é para iOS.

Você simplesmente tem que selecionar o widget que você quer e depois ir a partir daí.

Também não há necessidade de você se preocupar com o fato de os widgets serem muito específicos.

Isso significa que, mesmo que você acabe implementando alguns Cupertino, os widgets projetados por Maduna vão acabar parecendo iguais em todos os dispositivos iOS e Android existentes.

Portanto, não há necessidade de você se preocupar com algo em seu aplicativo ou até mesmo com o aplicativo como um todo, parecendo diferente.

Quando executado em dispositivos diferentes No entanto, você pode usar diferentes equipes Android e iOS, se desejar.

Também tudo é um widget em flertar.

Sim, até mesmo a sua classe de aplicativo, o aplicativo de material é um widget.

Então, toda a sua estrutura de layouts é escaneada por dimensão.

Então, ter tudo como um widget permite que você use flirter para criar UI impressionante de uma maneira muito simples.

Numero tres.

Vários pacotes, apesar de terem sido lançados mortos.

Recentemente, em 2007, a comunidade deflectiva de Dean é muito ativa e envolvida em torná-la melhor.

Isso fez com que o flatter fosse capaz de suportar muitos pacotes que você tem acesso a pacotes para abrir imagens compartilhando a corrente e fazer solicitações de HDTV acessando sensores armazenando suas preferências.

Implementando Firebase e a lista continua.

Todo esse suporte é para iOS e Android.

Agora que você passou por cima dos benefícios do sluttier, pergunte a si mesmo. Você está interessado em usar o flirter?

Não há necessidade de tomar uma decisão precipitada.

Estarei falando mais sobre tudo isso enquanto o discurso continua.


Seção 4: Como instalar o Flutter?
6. Instalando o Flutter para Windows, Linux e MacOS

Número um dos seus sistemas operacionais deve ser o Windows 7 S-B 1 são mais tarde de 64 bits, você precisa de pelo menos 400 M-B ou três espaço em disco.

Tome nota que não inclui este lugar para Dools.

Falando sobre as ferramentas que você precisa.

Ponto de shellfire Bovver.

Ou você não precisa.

Este é o G ID para o Windows.

Quando eles usam boa opção de prompt de comando do Windows.

Se você tiver para o Windows já está instalado.

Precisa ter certeza de que você pode executar o comando get para o prompt de comando ou instalar o flirter no Mac

SO para instalar e executar o flirter no Mac OS, seu ambiente de desenvolvimento deve atender a esses requisitos mínimos.

Estes são o sistema operacional deve ser, pelo menos, Mac OS 64 bits, pelo menos, 700 M-B de espaço livre em disco.

Novamente, isso não inclui espaço em disco para Dools.

Agora, observe que o flertador depende de várias ferramentas de linha de comando disponíveis em seu ambiente.

Estes são Bash e Diyar r m.

Boa menina.

Descompacte e até mesmo a árvore numérica.

Continuando a instalar o flirter no Linux para instalar e executar o flirter no Linux.

Você precisa de um sistema operacional que seja o Linux 64 bit 600 M-B de espaço livre em disco.

Isso não inclui espaço em disco para o seletor Dools, dependendo de algumas ferramentas de linha de comando disponíveis

em seu ambiente.

Estes são novamente Bash e ganham Diyar.

R m.

Boa menina.

Descompacte e qual comando de teste flexor depende da biblioteca estar disponível em seu ambiente que

é como você pode ver lib G.O. você Daut s o dot um fornecido por mim de bagagem, por exemplo l AB lib Glou one me no Ubuntu são Debian.

Você pode vê-lo no slide, então não deixe de reservar o site oficial do seletor.

Você tem uma ideia mais aprofundada sobre o procedimento que precisa ser seguido para instalar e começar a usar o flexor.



Seção 5: Benefícios do Flutter
7. Benefícios do Flutter

Com base em uma parte do Curso onde eu falei sobre o motivo pelo qual você deve considerar o uso do flirter um pouco mais sobre todos os benefícios e pode oferecer a você muitos desenvolvedores recomendam usar o flirter.

Se você está apenas começando no campo de desenvolvimento de aplicativos por causa de sua natureza amigável quando se trata de criar aplicativos multiplataforma, vamos passar por cima de uma lista.

Flirter de projeto de código aberto número um é um projeto de código aberto que permite que ele esteja disponível para uso também por startups.

Ele oferece muitas opções para criar o tipo de aplicativo que você deseja criar para Android ou iOS.

Além disso, sendo de código aberto, dá origem à colaboração aberta, permitindo que outros ajudem a melhorar ou faça o número dois de Ed impressionantes. integração usando flirter.

Você pode continuar e continuar adicionando e subtraindo edições durante o processo de desenvolvimento do aplicativo.

Oferece uma impressionante integração de edição com o estúdio Android e o código do Visual Studio.

Isso leva a conclusões mais inteligentes baseadas em onde os tipos de definições usuais e o número de módulos importados criam campos escuros como o Java.

Por que a dieta dietética não é uma cópia direta do Java.

Os dois são semelhantes a similaridades.

Ajude os desenvolvedores a mudar para o escuro.

Se eles estão usando Java para o número quatro da lista, temos 4 gerentes de engenharia que gerentes de engenharia também podem usar o flirter se tiverem a necessidade de liderar mais equipes de desenvolvimento biológico usando um SDK como gerente de engenharia.

Você pode criar uma equipe surda de um único aplicativo móvel ou uma equipe de desenvolvimento para ajudar você a unificar os investimentos em desenvolvimento.

Você pode enviar recursos mais rapidamente para reduzir seus custos de manutenção e transferir os mesmos recursos para o Android e o iOS simultaneamente.

Em suma, quando você está usando a desordem, você tem acesso a melhores recursos de widgets fáceis de usar e integrações editoriais impressionantes para o desenvolvimento de aplicativos.

Não é de admirar que muitos desenvolvedores tenham usado o seletor para criar aplicativos móveis, mas também para alcançar um público mais amplo.

No Android e no iOS.

Seção 6: Quanta Experiência como Desenvolvedor de Aplicativos
Devo ter que usar o Flutter?

8. Quanta experiência como desenvolvedor de aplicativos eu deveria
tem que usar o Flutter?

Esta é a pergunta que muitos de vocês podem ter atualmente.

Em suma, você não precisa de nenhuma experiência anterior para aprender e usar flirter.

Este SDK é muito acessível para programadores que têm uma ideia sobre conceitos de programação imperativa e conceitos orientados a objetos.

Contanto que você tenha paixão e interesse em usar a desordem para criar aplicativos, você achará o processo bastante fácil e rápido.

Vamos falar sobre isso um pouco mais.

Número um do estilo reativo ao usar o flirter, você precisará começar em algum lugar para criar um aplicativo para iOS e Android.

É por isso que você precisa estar pronto para trabalhar usando uma estrutura reativa.

Número fazer.

David você não pode falar sobre seletor sem falar sobre a arquitetura de widgets que oferece.

Como mencionado tudo é o widget e seletor de seu elemento de estrutura para eles e você é o botão dois elemento muito estilístico que você deseja usar, como um esquema de cores e formatar tudo.

É um isolante rígido que permite criar o tipo de aplicativo que você quer sem problemas.

Você ganha muito controle sobre a renderização do objeto de composição de execução do aplicativo e mais número três.

Diga adeus ao gerenciamento do ciclo de vida da atividade.

Se você é alguém que se envolveu um pouco no desenvolvimento de aplicativos, você estará familiarizado com o gerenciamento do ciclo de vida da atividade.

Ele só aparece como trabalho extra, não é?

Uma atividade periférica, se possível, que não seja adicionada diretamente no processo geral de produção de seu aplicativo.

No entanto, o que flertar você não terá que se preocupar com o gerenciamento do ciclo de vida da atividade.

Isso ocorre porque o patch do conector de desordem ajuda a fazer com que todas as atividades transmitidas sejam vestidas e carregadas de maneira síncrona.

Número de 60 SPSS consistentes são quadros por segundo.

O Flirter também é útil para quem procura criar um seletor de aplicativos gráficos ou de jogos pesados, capaz de oferecer uma taxa consistente de 60 quadros por segundo devido à natureza reativa e ao uso de suporte sombrio.

Você terá visuais estáveis ??em seu aplicativo desenvolvido.

O número cinco, o flertador de suporte da comunidade, também tem uma grande comunidade de expedientes, além de demitir desenvolvedores que podem ajudá-lo com o suporte necessário que você precisa.

A comunidade torna mais fácil para você aprender usando o flerte, bem como fazer conexões úteis no campo de desenvolvimento de aplicativos.

Então, vamos falar um pouco sobre darte e por que Slichter usa o começo.

Seção 7: Conhecendo o Dart
9. O que é o Dart e por que usá-lo?

Durante este curso eu mencionei d'arte muito como uma pequena atualização.

Dart é uma linguagem de programação de uso geral que foi originalmente desenvolvida pelo Google.

Mais tarde foi aprovado como padrão pela ECMA daks ECMA para 8 para ser mais preciso.

Você pode usar d'arte para construir aplicações web e móveis para servidores.

Então, por que os desenvolvedores do d'arte trabalham no Google, além de outros.

Não use darte para criar aplicativos de alta qualidade para iOS.

E Android, assim como a web.

Então é por isso que pode ser visto ou é visto como uma ótima opção quando se trabalha com recursos voltados para o desenvolvimento do lado do cliente.

Vamos a uma lista sobre o Bart.

A sintaxe escura produtiva número um é clara e concisa.

O ferramental geral é muito simples, mas poderoso.

Você pode identificar erros sutis precocemente através da digitação de sons usando darte.

Você também pode obter inúmeras bibliotecas de pontuação e milhares de pacotes.

Número dois rápido Dart pode ser denominado como uma linguagem que oferece otimização antes da violação.

Isso ajuda você a obter desempenho previsivelmente alto e inicialização rápida em dispositivos móveis e na Web.

O número três viável para aplicativos móveis escuros é executado de forma nativa no Android e no iOS.

Compiladores sombrios fazem A-R como um código X 86.

Tome nota DEC para web app caiu spiles para javascript e é acessível.

Muitos desenvolvedores estão familiarizados com o d'arte que você faz.

Não é uma orientação e sintaxe de objeto sem surpresas.

Você pode facilmente pegar o jeito escuro se você já sabe sobre Java são C ++.

Agora, se você sabe um pouco mais sobre d'arte Lexi, por que você deve considerar aprendê-lo em 2018, bem como por que o flerte faz uso do efeito.

Seção 8: Por que você deve considerar o aprendizado de dardo em
2018

10. Se você considerar a aprendizagem do dardo em 2018 e
Além?

Se você tem pensado em aprender, aqui está uma lista que pode ajudá-lo a tomar uma decisão.

Número um fácil de aprender.

Como mencionado anteriormente, os guardas de Daryn compartilham semelhanças com outras linguagens comuns.

Isso também significa que você já deve saber como usar d'arte sem perceber que será fácil para você aprender se tiver experiência com outras linguagens orientadas a objetos, como Java, são C ++.

Mesmo se você souber alguma aprendizagem sobre javascript, Darb não será difícil para você.

Outra grande coisa é que o dardo suporta tanto a digitação solta quanto a forte.

Isso permite que seja mais fácil quando você está mudando de outro idioma.

Number faz uma base de código compartilhada nativamente compilada.

Alguns de vocês podem saber que outras estruturas permitem que você compartilhe partes da base de código em diferentes plataformas.

No entanto escuro é um pouco diferente.

Ele permite que você escreva um único aplicativo que pode ser usado tanto no iOS quanto no Android, juntamente com a maneira como ele é compilado nativamente pelo Dubost.

Agora, é claro, isso é algo que os outros podem fazer, mas o darte é capaz de permitir ainda mais o compartilhamento.

Você sabe que eu posso dizer que permite que você compartilhe um pouco do seu código entre a inteireza do seu GUI, não apenas os aplicativos móveis fazem o Menke para diferentes plataformas, significando iOS e Android, mas também para a Web, bem como outras aplicações de guarda também .

Isso permite que você compartilhe a maior parte do código não UI ou digamos que o código específico em preto com o número de frameworks do Underdark seja produtivo. Usando o escuro, você terá a experiência de como é fácil e rápido criar o layout e adicionar funcionalidade para o seu aplicativo.

Você pode criar o Caly no código criando uma série de widgets.

Lembre-se do que eu disse a você que, em flertar, tudo o que o widget usa com essas ferramentas certamente ajudará você a criar o que quiser no menor tempo possível.

Número quatro compila antes do tempo e apenas no tempo quando você está desenvolvendo um aplicativo Avodart você pode ver as mudanças que você faz instantaneamente.

Isso significa que você não terá que passar pelo procedimento de recompilação e não há necessidade de esperar que o aplicativo seja recarregado.

Você só precisa salvar a alteração e ver instantaneamente as alterações.

Isso é possível porque, com essa estrutura de desenvolvimento de aplicativo para celular, você pode compilar antecipadamente.

R E ou D, bem como na hora certa.

R d.

Agora que você está ciente das razões mencionadas, você provavelmente está começando a entender o benefício de usar o darte.

Experimente e veja se é algo sobre o qual você deseja aprender mais.

Como um desenvolvedor de aplicativos que está mudando meu discurso, eu deixo você saber por que flirter escolheu usar d'arte.

Seção 9: Por que o Flutter usa Dart?

11. Por que o Flutter Decidiu Usar o Dart?
Flirter decidir usar d'arte não foi uma decisão precipitada.

Flirter usa essa ferramenta.

Um total de quatro dimensões primárias para avaliação e considerou todas as necessidades dos autores, desenvolvedores e usuários finais do framework.

Foi fundada enquanto algumas línguas cumpriam certos requisitos.

Estava escuro lá.

Marcou muito em todas as dimensões de avaliação que foram verificadas pela equipe mais plana e também atendeu a todos os requisitos e critérios determinados.

Você deve saber que os dubladores de Darkthrone e compiladores suportam a combinação de dois recursos críticos para flertar um parding ao tema de flutter.

O primeiro é o ciclo de desenvolvimento rápido baseado em GAAP, que permite a mudança de forma e a carga de Hawtry com estado na linguagem com tipos.

Como eu mencionei anteriormente, o GAAP está na hora certa.

A segunda é a antecipação de nosso descompilador imenso, que emite uma importação A-R eficiente para uma inicialização rápida e desempenho previsível de implementações de produção.

Além disso, a equipe de flirter foi capaz de trabalhar de perto com a comunidade DAR.

A comunidade é ativa e continua a investir recursos para melhorar o desenho quando se trata de usá-lo como exemplo.

Quando a equipe adotada darte decidiu definhar não teve uma hora antes do tempo ou toolbol AOB Freberg usando binário nativo é um tal recurso é fundamental para alcançar alto desempenho previsível.

No entanto, agora a linguagem tem isso porque eles são dark deam e eles estão construindo para Lucker no dia primário um grande dia com dark marcou uma pontuação alta para a vibração deam foram número um produtividade do desenvolvedor de acordo com o tema.

Uma das principais proposições de valor de letras é que ele economiza recursos de engenharia, permitindo que desenvolvedores criem aplicativos para iOS e Android com a mesma base de código usando uma linguagem altamente produtiva que acelera os desenvolvedores Forder e torna a carta mais atraente para eles.

Isso foi muito importante para a equipe de framework defletor da Dubow, bem como para os desenvolvedores.

Você deve saber que a maior parte da sorte é construída na mesma linguagem dada aos usuários.

Portanto, o tema precisa permanecer produtivo em cem mil linhas de código sem sacrificar a acessibilidade ou a legibilidade da estrutura e dos widgets para os desenvolvedores.

Orientação número dois objeto para flirter que vorn tema fez uma linguagem que foi adequado para flusters domínio do problema que é bom outra coisa que você vai usar suas experiências de acordo com a equipe da indústria tem várias décadas de conveniência construindo frameworks de interface do usuário em linguagens orientadas a objeto.

Embora eles possam ter usado uma linguagem conhecida orientada a objetos, isso significaria reinventar a roda para resolver vários problemas difíceis.

Além disso, a grande maioria dos desenvolvedores já tem experiência com desenvolvimento orientado a objetos, tornando mais fácil aprender como desenvolver seus aplicativos, cujo seletor agora possui um alto desempenho previsível de acordo com o refletor de temas.

Eles querem capacitar os desenvolvedores para criar experiências de usuário rápidas e fluidas.

Então, para que eles cumpram essa promessa ou objetivo, eles precisam ser capazes de executar uma quantidade significativa de placa de desenvolvedor durante cada quadro de animação.

Isso significa que a equipe precisava de uma linguagem que pudesse fornecer alto desempenho e fornecer desempenho previsível sem bater seus chefes.

Isso poderia causar o número de quadros perdidos para localização fasc neste ponto.

Você deve saber que a estrutura do defletor utiliza um fluxo de estilo funcional que, de acordo com o tema da letra, depende muito do eficiente alocador de memória subjacente que manipula com eficiência o pequeno local de curta duração.

Este arquivo foi desenvolvido em linguagens com essa propriedade e não funciona de maneira eficiente em idiomas como esse recurso.

Então agora você sabe por que flutter decidiu usar d'arte e por que você deveria estar familiarizado com tal linguagem.

E quem sabe você pode querer usá-lo também.


Seção 10: Perguntas Freqüentes sobre o Flutter
12. O que há dentro do Flutter SDK?
Permita-me ir com freqüência para fazer perguntas que você possa ter em mente que desapontam.

A primeira pergunta é o que está dentro do SDK flertador.

Quando se trata de responder o que está dentro do deflector SDK.

Aqui está a lista.

O número um otimizou fortemente o primeiro mecanismo de renderização de leads duplo da Mumbai com excelente suporte para texto.

Em seguida, temos o conjunto de widgets de estilo de re-ato de Morgan o conjunto rico de widgets para Android e iOS que eu disse anteriormente tudo é um widget em flertar.

Número quatro, é bom para testes de unidade e integração.

Em seguida, entramos em interoperabilidade e plug-in API para conectar-se ao sistema e participar D DST gais headless best runner para executar testes em ferramentas de linha de comando do Windows Linux e Mac para criar testes de construção e compilar seus aplicativos.

13. O Flutter vem com uma estrutura?

Faz flutter vem com o quadro.

Sim.

Como desenvolvedores de aplicativos, você deve saber que os sistemas de terra plana com uma estrutura moderna inspirada pela estrutura de aletas de re-ação foram projetados para serem lidos e personalizáveis ??como um desenvolvedor.

Você pode ir em frente e optar por usar apenas parte do framework que você tem acesso, você é um framework completamente diferente.

A escolha é sua.


14. O Flutter vem com widgets?
Faz flutter vem com Vineyard como mencionado antes que sacode no parque e parte de flutter.

Então, sim, selecione seus navios com um conjunto de alta qualidade McKeel design em layouts e equipes de widgets Cupertino.

Você também pode ir em frente e fazer seus próprios widgets ou até mesmo personalizar qualquer um dos widgets existentes para criar o tipo de aplicativo que você deseja.

Novamente lembre-se tudo é um widget e flutter.


15. Como o Flutter roda meu código no Android?

Como flertar meu no Android.

Alguns de vocês podem estar se perguntando como o Slichter executa seu código e o direito de lhe dar uma resposta simples.

De acordo com a equipe do flooder, o código Engine C e C ++ são compilados com andróides e DK The Dark Lord.

Boada SDK do que o seu está à frente do tempo.

AOP compilado em uma biblioteca ARMM nativa.

Essa biblioteca está incluída em um projeto de corredor e dreich e a coisa toda é construída em qualquer B.K. quando é lançado, o



\chapter{Material e Métodos} \label{ch:meto}

Como apresentado no capítulo \ref{ch:fund}, muitos trabalhos com imagens são feitos na tentativa de criar uma ferramenta que auxilie a etapa de análise visual do teste de tetrazólio, analisando os padrões característicos dos danos e definindo quais danos cada sementes possuí.


Foi elaborado e apresentado um projeto de pesquisa ao comitê de ética da Universidade Federal Fluminense (UFF), para a obtenção de imagens térmicas por câmera sensível a radiação infravermelha nos pacientes (que consentisse em participar da pesquisa) do Ambulatório de Ginecologia do Hospital Universitário Antônio Pedro (HUAP), e suas disponibilizações para pesquisa. O projeto foi aprovado pelo Comitê de Ética em Pesquisa (CEP) em 04/06/2012 e está registrado na Plataforma Brasil, sob o número CAAE 01042812.0.0000.5243, do Ministério da Saúde. Neste projeto colaboram diversos mastologistas e radiologistas do HUAP. O objetivo é verificar a eficiência das imagens térmicas no diagnóstico precoce de doenças da mama, principalmente o câncer. Para isso, um passo preliminar é construir um banco de dados que permita gerenciar e recuperar informações dos exames dos pacientes, que é o motivo de esta dissertação. O banco de dados servirá para auxiliar em pesquisas futuras sobre imagens térmicas de nosso grupo e de outros grupos de pesquisa.



\section{Captura de imagens}

O problema das metodologias usadas anteriormente é o processo de captura das imagens. A maioria dos métodos exigem a maior padronização possível das imagens para o processamento digital, exigindo uma estrutura para fotografar as sementes, controlando a iluminação, sombra, foco, distancia, fundo de fácil identificação e a captura da imagem da feição interior e exterior da semente. E levando em consideração que no teste de tetrazólio, para análise de um único lote de soja, é necessário o corte e observação de cem sementes, o processo de captura e catalogação dos dados se torna demorado e pouco produtivo. Sendo este o motivo pelo qual os trabalhos citados utilizam entre xxx e yyy imagens.

Metodologias de análise de imagens como \ew{deep learning} não foram usadas até o momento pela dificuldade de catalogar uma grande quantidade de imagem exigias, na ordem de 20.000 para cada classe. A base de informações deste trabalho, foi projetada para ter no mínimo 80.000 imagens, sendo 20.000 de cada um dos três principais danos encontrados - umidade, percevejo e mecânico -, e 20.000 imagens de sementes sem danos.


\section{Analise especialista}

\nota{Aqui o objetivo é ressaltar que o processo de coleta de dados para pesquisa não deve onerar a atividade de análise}

Durante a catalogação das imagens é preciso que um especialista classifique os danos de cada semente e vincule esta informação com cada imagem, este processo é uma etapa a mais na rotina de análise, pois atualmente o analista marca em uma ficha de papel quais danos foram detectados, porém sem indicar em qual sementes, pois para o resultado, basta a quantidade total dos danos por classe. Cada etapa que aumente o tempo análise representar um aumento no custo de produção da soja. Para estimar o tempo usado e a quantidade de análises de tetrazólio de uma safra é preciso calcular o número de lotes de sementes. 

Segundo \cite{FAS2018}, na safra 2013/2014 \nota{Vou atualizar os valores para safra de 2017} no Brasil foram plantados 30,17 Milhões de Hectares de soja. Para calcular a quantidade total de lotes criados para atender esta demanda de plantio é preciso saber a participação de cada cultivar de soja na produção total. Apesar de não existirem informações divulgadas sobre estes valores é possível usar o valor médio para fazer uma estimativa. Para tal foi criada a seguinte fórmula: \nota{Aqui a tentativa é calcular quantas horas de trabalho de análise foram gastas, no Brasil em uma safra, com o teste de tetrazólio}

\begin{equation}
L = \frac{H.P.M}{G. 4,8.10^{+7}} 
\end{equation}

Onde,
L é o total de lotes necessários.
H é o total de hectares plantados.
P é a população de plantas desejadas por hectare.
M é o peso em gramas de mil sementes (PSM).
G é o percentual de germinação.

A constante 4,8E+7 é calculada considerando que um lote é formado por 120 sacas de 40 Kg, o valor recomendado por \cite{abc} de P é de 320.000, 80\% de germinação é o mínimo aceito por lei por tanto será usado 90\% para G. Porém determinar M, o peso de 1000 sementes, é um desafio maior pois não basta encontrar a média registrada de cada cultivar, pois a falta de chuva durante o período de enchimento de grão pode resultar em sementes pequenas e leves \cite{abc} conhecido como chumbinho, que apesar de não possuir o tamanho e peso tradicional da semente tem o mesmo poder germinativo. \cite{abc} Mostra que a média de PSM encontrada no Brasil é de 175 gramas. Logo, estima-se que foi necessário para a safra Brasileira de soja 2013/2014 a análise de 391.133 lotes de sementes. Cada análise de tz de soja deve ser feita a avaliação uma a uma de 100 sementes por lote, que seriam 39.113.300 de sementes analisadas. \cite{abc} Recomenda que a empresa produtora de sementes realize uma análise no recebimento do lote e outra no final do beneficiamento, apesar de algumas empresas realizarem apenas uma vez o teste, outras realizam varias vezes para acompanhar a situação dos lotes. 


Além das oito horas de preparação da amostra, um analista treinado e experiente consegue realizar cinco análises por hora, ao passo de um analista pouco treinado realiza duas análises, chegando assim na média de 3,5 análises por hora, o que representa 223.504 horas de trabalho de análise. Desta forma seria necessário 115 analistas de sementes trabalhando oito horas por dia durante um ano para atender esta demanda. \nota{Este cálculo foi criado por mim com ajuda de um pesquisador da Embrapa em 2014. No trabalho final, esta parte foi retirada pela limitação de páginas para submissão. Eu gostaria de incluir nesta dissertação para contribuir com a direção que pretendo dar ao texto, que é um processo demorado e delicado e que se beneficiará no uso de uma ferramenta computacional. Não sei se este cálculo se encaixa na metodologia deste trabalho, então penso que eu poderia fazer um artigo menor com ele e referenciá-lo. Aproveitando para fazer um pouco da pontuação
}


\section{Validação da análise}
No procedimento atual de análise de sementes, quando uma amostra é retirada de um lote, são coletadas uma quantidade superior da necessária para posterior conferência em caso de fiscalizações e auditorias, porém, é realizado um novo teste, em outras cem sementes da mesma amostra, pois o processo de análise acaba destruindo a semente.

Uma rotina de análise que registre a imagem das sementes utilizadas promove um processo mais transparente, facilitando as fiscalizações e diminuindo a necessidade de refazer os testes.


\section{Estratégia de uso} \nota{Aqui pretendo apresentar a proposta de solução que atenda as dificuldades apresentadas anteriormente}

Para que uma proposta de sistema de catalogação tenha sucesso, deve ser aderente ao fluxo de trabalho atual dos analistas e oferecer benefícios pelo seu uso. Para tal, este trabalho se propõe a desenvolver uma ferramenta prática para o uso durante as análises de tetrazólio enquanto coleta e cataloga os dados para posteriores estudos. 


Algumas coisas que ainda preciso escrever ...

horas de trabalho

aderência ao processo - rapidez

base de imagens - volume de imagens

padronização
gabarito

vantagens - Relatório
auditoria

aplicativo de celular

validação dos dados






%\begin{center}
	\begin{longtable}{|l|l|}
		
		\hline
		
		{CameraCaptureSession} & {A configured capture session for a CameraDevice, used for capturing images from the camera or reprocessing images captured from the camera in the same session previously.}\\ \hline
		{CameraCaptureSession.CaptureCallback} & {callback object for tracking the progress of a CaptureRequest submitted to the camera device.}\\ \hline
		{CameraCaptureSession.StateCallback} & {A callback object for receiving updates about the state of a camera capture session.}\\ \hline
		{CameraCharacteristics} & {The properties describing a CameraDevice.}\\ \hline	
		{CameraCharacteristics.Key\textless T\textgreater} & {A Key is used to do camera characteristics field lookups with CameraCharacteristics.get(CameraCharacteristics.Key).}\\ \hline	
		{CameraConstrainedHighSpeedCaptureSession} & {A constrained high speed capture session for a CameraDevice, used for capturing high speed images from the CameraDevice for high speed video recording use case.}\\ \hline
		{CameraDevice} & {The CameraDevice class is a representation of a single camera connected to an Android device, allowing for fine-grain control of image capture and post-processing at high frame rates.}\\ \hline
		{CameraDevice.StateCallback} & {A callback objects for receiving updates about the state of a camera device.}\\ \hline
		{CameraManager} & {A system service manager for detecting, characterizing, and connecting to CameraDevices.}\\ \hline
		{CameraManager.AvailabilityCallback} & {A callback for camera devices becoming available or unavailable to open.}\\ \hline
		{CameraManager.TorchCallback} & {A callback for camera flash torch modes becoming unavailable, disabled, or enabled.}\\ \hline
		{CameraMetadata\textless TKey\textgreater} & {The base class for camera controls and information.}\\ \hline
		{CaptureFailure} & {A report of failed capture for a single image capture from the image sensor.}\\ \hline
		{CaptureRequest} & {An immutable package of settings and outputs needed to capture a single image from the camera device.}\\ \hline
		{CaptureRequest.Builder} & {A builder for capture requests.}\\ \hline
		{CaptureRequest.Key\textless T\textgreater} & {A Key is used to do capture request field lookups with CaptureResult.get(CaptureResult.Key) or to set fields with CaptureRequest.Builder.set(Key, Object).}\\ \hline
		{CaptureResult} & {The subset of the results of a single image capture from the image sensor.}\\ \hline
		{CaptureResult.Key\textless T\textgreater} & {A Key is used to do capture result field lookups with CaptureResult.get(CaptureResult.Key).}\\ \hline
		{DngCreator} & {The DngCreator class provides functions to write raw pixel data as a DNG file.}\\ \hline
		{TotalCaptureResult} & {The total assembled results of a single image capture from the image sensor.}\\ \hline
		
	\end{longtable}
\end{center}






\begin{center}
	\begin{longtable}{|l|l|l|}
		\caption[Feasible triples for a highly variable Grid]{Feasible triples for 
			highly variable Grid, MLMMH.} \label{grid_mlmmh} \\
		
		\hline \multicolumn{1}{|c|}{\textbf{Time (s)}} & \multicolumn{1}{c|}{\textbf{Triple chosen}} & \multicolumn{1}{c|}{\textbf{Other feasible triples}} \\ \hline 
		\endfirsthead
		
		\multicolumn{3}{c}%
		{{\bfseries \tablename\ \thetable{} -- continued from previous page}} \\
		\hline \multicolumn{1}{|c|}{\textbf{Time (s)}} &
		\multicolumn{1}{c|}{\textbf{Triple chosen}} &
		\multicolumn{1}{c|}{\textbf{Other feasible triples}} \\ \hline 
		\endhead
		
		\hline \multicolumn{3}{|r|}{{Continued on next page}} \\ \hline
		\endfoot
		
		\hline \hline
		\endlastfoot
		
		0 & (1, 11, 13725) & (1, 12, 10980), (1, 13, 8235), (2, 2, 0), (3, 1, 0) \\
		2745 & (1, 12, 10980) & (1, 13, 8235), (2, 2, 0), (2, 3, 0), (3, 1, 0) \\
		5490 & (1, 12, 13725) & (2, 2, 2745), (2, 3, 0), (3, 1, 0) \\
		8235 & (1, 12, 16470) & (1, 13, 13725), (2, 2, 2745), (2, 3, 0), (3, 1, 0) \\
		10980 & (1, 12, 16470) & (1, 13, 13725), (2, 2, 2745), (2, 3, 0), (3, 1, 0) \\
		13725 & (1, 12, 16470) & (1, 13, 13725), (2, 2, 2745), (2, 3, 0), (3, 1, 0) \\
		16470 & (1, 13, 16470) & (2, 2, 2745), (2, 3, 0), (3, 1, 0) \\
		19215 & (1, 12, 16470) & (1, 13, 13725), (2, 2, 2745), (2, 3, 0), (3, 1, 0) \\
		21960 & (1, 12, 16470) & (1, 13, 13725), (2, 2, 2745), (2, 3, 0), (3, 1, 0) \\
		24705 & (1, 12, 16470) & (1, 13, 13725), (2, 2, 2745), (2, 3, 0), (3, 1, 0) \\
		27450 & (1, 12, 16470) & (1, 13, 13725), (2, 2, 2745), (2, 3, 0), (3, 1, 0) \\
		30195 & (2, 2, 2745) & (2, 3, 0), (3, 1, 0) \\
		32940 & (1, 13, 16470) & (2, 2, 2745), (2, 3, 0), (3, 1, 0) \\
		35685 & (1, 13, 13725) & (2, 2, 2745), (2, 3, 0), (3, 1, 0) \\
		38430 & (1, 13, 10980) & (2, 2, 2745), (2, 3, 0), (3, 1, 0) \\
		41175 & (1, 12, 13725) & (1, 13, 10980), (2, 2, 2745), (2, 3, 0), (3, 1, 0) \\
		43920 & (1, 13, 10980) & (2, 2, 2745), (2, 3, 0), (3, 1, 0) \\
		46665 & (2, 2, 2745) & (2, 3, 0), (3, 1, 0) \\
		49410 & (2, 2, 2745) & (2, 3, 0), (3, 1, 0) \\
		52155 & (1, 12, 16470) & (1, 13, 13725), (2, 2, 2745), (2, 3, 0), (3, 1, 0) \\
		54900 & (1, 13, 13725) & (2, 2, 2745), (2, 3, 0), (3, 1, 0) \\
		57645 & (1, 13, 13725) & (2, 2, 2745), (2, 3, 0), (3, 1, 0) \\
		60390 & (1, 12, 13725) & (2, 2, 2745), (2, 3, 0), (3, 1, 0) \\
		63135 & (1, 13, 16470) & (2, 2, 2745), (2, 3, 0), (3, 1, 0) \\
		65880 & (1, 13, 16470) & (2, 2, 2745), (2, 3, 0), (3, 1, 0) \\
		68625 & (2, 2, 2745) & (2, 3, 0), (3, 1, 0) \\
		71370 & (1, 13, 13725) & (2, 2, 2745), (2, 3, 0), (3, 1, 0) \\
		74115 & (1, 12, 13725) & (2, 2, 2745), (2, 3, 0), (3, 1, 0) \\
		76860 & (1, 13, 13725) & (2, 2, 2745), (2, 3, 0), (3, 1, 0) \\
		79605 & (1, 13, 13725) & (2, 2, 2745), (2, 3, 0), (3, 1, 0) \\
		82350 & (1, 12, 13725) & (2, 2, 2745), (2, 3, 0), (3, 1, 0) \\
		85095 & (1, 12, 13725) & (1, 13, 10980), (2, 2, 2745), (2, 3, 0), (3, 1, 0) \\
		87840 & (1, 13, 16470) & (2, 2, 2745), (2, 3, 0), (3, 1, 0) \\
		90585 & (1, 13, 16470) & (2, 2, 2745), (2, 3, 0), (3, 1, 0) \\
		93330 & (1, 13, 13725) & (2, 2, 2745), (2, 3, 0), (3, 1, 0) \\
		96075 & (1, 13, 16470) & (2, 2, 2745), (2, 3, 0), (3, 1, 0) \\
		98820 & (1, 13, 16470) & (2, 2, 2745), (2, 3, 0), (3, 1, 0) \\
		101565 & (1, 13, 13725) & (2, 2, 2745), (2, 3, 0), (3, 1, 0) \\
		104310 & (1, 13, 16470) & (2, 2, 2745), (2, 3, 0), (3, 1, 0) \\
		107055 & (1, 13, 13725) & (2, 2, 2745), (2, 3, 0), (3, 1, 0) \\
		109800 & (1, 13, 13725) & (2, 2, 2745), (2, 3, 0), (3, 1, 0) \\
		112545 & (1, 12, 16470) & (1, 13, 13725), (2, 2, 2745), (2, 3, 0), (3, 1, 0) \\
		115290 & (1, 13, 16470) & (2, 2, 2745), (2, 3, 0), (3, 1, 0) \\
		118035 & (1, 13, 13725) & (2, 2, 2745), (2, 3, 0), (3, 1, 0) \\
		120780 & (1, 13, 16470) & (2, 2, 2745), (2, 3, 0), (3, 1, 0) \\
		123525 & (1, 13, 13725) & (2, 2, 2745), (2, 3, 0), (3, 1, 0) \\
		126270 & (1, 12, 16470) & (1, 13, 13725), (2, 2, 2745), (2, 3, 0), (3, 1, 0) \\
		129015 & (2, 2, 2745) & (2, 3, 0), (3, 1, 0) \\
		131760 & (2, 2, 2745) & (2, 3, 0), (3, 1, 0) \\
		134505 & (1, 13, 16470) & (2, 2, 2745), (2, 3, 0), (3, 1, 0) \\
		137250 & (1, 13, 13725) & (2, 2, 2745), (2, 3, 0), (3, 1, 0) \\
		139995 & (2, 2, 2745) & (2, 3, 0), (3, 1, 0) \\
		142740 & (2, 2, 2745) & (2, 3, 0), (3, 1, 0) \\
		145485 & (1, 12, 16470) & (1, 13, 13725), (2, 2, 2745), (2, 3, 0), (3, 1, 0) \\
		148230 & (2, 2, 2745) & (2, 3, 0), (3, 1, 0) \\
		150975 & (1, 13, 16470) & (2, 2, 2745), (2, 3, 0), (3, 1, 0) \\
		153720 & (1, 12, 13725) & (2, 2, 2745), (2, 3, 0), (3, 1, 0) \\
		156465 & (1, 13, 13725) & (2, 2, 2745), (2, 3, 0), (3, 1, 0) \\
		159210 & (1, 13, 13725) & (2, 2, 2745), (2, 3, 0), (3, 1, 0) \\
		161955 & (1, 13, 16470) & (2, 2, 2745), (2, 3, 0), (3, 1, 0) \\
		164700 & (1, 13, 13725) & (2, 2, 2745), (2, 3, 0), (3, 1, 0) \\
	\end{longtable}
\end{center}



\begin{table}[h]

	\begin{tabularx}{\textwidth}{|l|X|}
	\hline
	
{CameraCaptureSession} & {A configured capture session for a CameraDevice, used for capturing images from the camera or reprocessing images captured from the camera in the same session previously.}\\ \hline
{CameraCaptureSession.CaptureCallback} & {callback object for tracking the progress of a CaptureRequest submitted to the camera device.}\\ \hline
{CameraCaptureSession.StateCallback} & {A callback object for receiving updates about the state of a camera capture session.}\\ \hline
{CameraCharacteristics} & {The properties describing a CameraDevice.}\\ \hline	
{CameraCharacteristics.Key\textless T\textgreater} & {A Key is used to do camera characteristics field lookups with CameraCharacteristics.get(CameraCharacteristics.Key).}\\ \hline	
{CameraConstrainedHighSpeedCaptureSession} & {A constrained high speed capture session for a CameraDevice, used for capturing high speed images from the CameraDevice for high speed video recording use case.}\\ \hline
{CameraDevice} & {The CameraDevice class is a representation of a single camera connected to an Android device, allowing for fine-grain control of image capture and post-processing at high frame rates.}\\ \hline
{CameraDevice.StateCallback} & {A callback objects for receiving updates about the state of a camera device.}\\ \hline
{CameraManager} & {A system service manager for detecting, characterizing, and connecting to CameraDevices.}\\ \hline
{CameraManager.AvailabilityCallback} & {A callback for camera devices becoming available or unavailable to open.}\\ \hline
{CameraManager.TorchCallback} & {A callback for camera flash torch modes becoming unavailable, disabled, or enabled.}\\ \hline
{CameraMetadata\textless TKey\textgreater} & {The base class for camera controls and information.}\\ \hline
{CaptureFailure} & {A report of failed capture for a single image capture from the image sensor.}\\ \hline
{CaptureRequest} & {An immutable package of settings and outputs needed to capture a single image from the camera device.}\\ \hline
{CaptureRequest.Builder} & {A builder for capture requests.}\\ \hline
{CaptureRequest.Key\textless T\textgreater} & {A Key is used to do capture request field lookups with CaptureResult.get(CaptureResult.Key) or to set fields with CaptureRequest.Builder.set(Key, Object).}\\ \hline
{CaptureResult} & {The subset of the results of a single image capture from the image sensor.}\\ \hline
{CaptureResult.Key\textless T\textgreater} & {A Key is used to do capture result field lookups with CaptureResult.get(CaptureResult.Key).}\\ \hline
{DngCreator} & {The DngCreator class provides functions to write raw pixel data as a DNG file.}\\ \hline
{TotalCaptureResult} & {The total assembled results of a single image capture from the image sensor.}\\ \hline
	
	\end{tabularx}
\end{table}

%\input{exper.tex}
%\input{concl.tex}


%
1:28
Objetivo é

%---------- Referencias ----------
\clearpage % this is need for add +1 to pageref of bibstart used in 'ficha catalografica'.
%\label{bibstart}
\bibliography{dissertacao_von} % geracao automatica das referencias a partir do arquivo bibliografia.bib%
%\label{bibend}


\end{document}
