

\begin{list}{label}{spacing}
	\item 
\end{list}


\chapter{Introdução} \label{ch:intro}

O cultivo da soja \lw{Glycine max (L.) Merr.} representa grande importância para o Brasil, promovendo desenvolvimento tecnológico, movimentando o mercado financeiro e a industria alimentícia. Ainda, interioriza o processo de urbanização, gerando emprego, renda e arrecadação de impostos.

Para garantir que o processo de cultivo seja eficiente, laboratórios de analise de sementes realizam diversos testes para mensurar as condições de qualidade dos lotes produzidos. Para avaliação do vigor de sementes de soja, recomenda-se os de envelhecimento acelerado, tetrazólio, condutividade elétrica, crescimento de plântulas, classificação do vigor de plântulas (Vieira et al., 2003). Destes, o teste de tetrazólio se destaca, principalmente para a soja, devido à sua rapidez, precisão e também pelo grande número de informações fornecidas pelo mesmo \cite{FrancaNeto1998}.

Na etapa de interpretação do teste de tetrazólio, o analista desfruta de seus conhecimentos da fisiologia de sementes, para classificar visualmente cada uma das cem sementes por lote individualmente. \citeonline{FrancaNeto1998} aponta que a precisão dos resultados depende diretamente da experiência, interpretação dos dados e julgamento crítico do analista. E por ser relativamente tedioso, uma vez que as sementes são avaliadas uma a uma, não é recomendado a realização de muitos testes no mesmo dia, para que o cansaço não atrapalhe o resultado da análise.

Em atividades que podem ser demoradas e estressantes, o uso de sistemas computacionais inteligentes pode automatizar processos proporciona resultados resultados melhores e mais confiáveis que o esperado para o ser humano (Lesk, 2008). Os sistemas que utilizam técnicas de Visão Computacional, são exemplos de sistemas que buscam assemelhar-se à visão humana para solução de problemas complexos como o reconhecimento de imagens.

Pesquisas como \citeonline{MECHELN} e \citeonline{MarcosFilho2009}, propõe o desenvolvimento de metodologias automáticas para realização da classificação de sementes de soja pelo teste de tetrazólio, utilizam técnicas de Visão Computacional. 

Como exposto, a realização de pesquisas para o desenvolvimento de sistemas computacionais para classificação de sementes, especialmente de tetrazólio, requerem fundamentalmente o uso de conjuntos de dados mais robusto e confiável. 
Diante deste contexto, a presente pesquisa propõe o desenvolvimento de metodologia de catalogação de imagens e captura de informações \lw{In loco} para uso em laboratórios de classificação de sementes.


\section{Objetivos}

Criar uma base catalográfica de imagens baseada em visão computacional, com informações associadas, de sementes de soja submetidas ao teste tetrazólio.


\begin{enumerate}
	\item Desenvolver um aplicativo para dispositivos móveis para coleta das imagens e classificação;
	\item Sincronizar as informações com um servidor centralizado;
	\item Gerar relatório para análise estatística dos dados;
	\item Gerar \ew{data sets} das informações;
\end{enumerate}

\section{Justificativa e contribuições}

Com a intenção de possibilitar trabalhos futuros em reconhecimento de padrões com visão computacional, que identifiquem os danos em sementes de soja auxiliando no processo de análise do teste de tetrazólio, este trabalho tem o objetivo de projetar, coletar e disponibilizar uma base de informações com imagens ampla e confiável.

A metodologia de desenvolvimento foi projetada para que o analista profissional possa fazer a coleta dos dados durante sua atividade de rotina sem prejudicar seu desempenho habitual. Além de fornecer uma ferramenta que continuará auxiliando-o mesmo após a finalização deste projeto de catalogação.

Esta metodologia promove benefícios diretos e imediatos aos envolvidos no processo de execução do teste de tetrazólio em sementes de soja: ao analista de sementes, ao laboratório responsável, a empresa que comercializa a as sementes, aos órgãos fiscalizadores e ao produtor que adquire o lote de semente. Ao desenvolvê-lá desta forma abriu-se as possibilidades de parcerias com laboratórios de sementes, na colaboração com equipamentos e mão de obra qualificada.

%Esta metodologia é descrita detalhadamente no capitulo \ref{ch:meto}.


\section{Organiza��o do trabalho}


A organiza��o da estrutura deste trabalho encontra-se dividida em seis cap�tulos, al�m das refer�ncias bibliogr�ficas. Neste primeiro cap�tulo � feita uma introdu��o do trabalho apresentando-se os objetivos a cumprir. 


No segundo cap�tulo s�o apresentados conceitos b�sicos sobre o teste de tetraz�lio e os principais danos detectados nas semente durante a an�lise.

O terceiro cap�tulo faz uma revis�o dos conceitos relacionados a sistemas CBIR e � literatura existente na �rea, descrevendo as t�cnicas mais utilizadas considerando sua rela��o com a �rea m�dica. 

O quarto cap�tulo apresenta, de forma detalhada, a metodologia proposta para modelagem do banco de dados e no quinto cap�tulo � descrito a metodologia de CBIR proposto. 

O sexto cap�tulo trata da apresenta��o de resultados da metodologia proposta, para tal fim s�o utilizadas imagens t�rmicas e mamografias. 
O �ltimo cap�tulo faz considera��es finais e prop�e id�ias para trabalhos futuros
