\section{Organização do trabalho}


A organização da estrutura deste trabalho encontra-se dividida em seis capítulos, além das referências bibliográficas. Neste primeiro capítulo é feita uma introdução do trabalho apresentando-se os objetivos a cumprir. 


No segundo capítulo são apresentados conceitos básicos sobre o teste de tetrazólio e os principais danos detectados nas semente durante a análise.

O terceiro capítulo faz uma revisão dos conceitos relacionados a sistemas CBIR e à literatura existente na área, descrevendo as técnicas mais utilizadas considerando sua relação com a área médica. 

O quarto capítulo apresenta, de forma detalhada, a metodologia proposta para modelagem do banco de dados e no quinto capítulo é descrito a metodologia de CBIR proposto. 

O sexto capítulo trata da apresentação de resultados da metodologia proposta, para tal fim são utilizadas imagens térmicas e mamografias. 
O último capítulo faz considerações finais e propõe idéias para trabalhos futuros