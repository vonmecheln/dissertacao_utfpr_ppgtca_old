\begin{resumo} 
	%	Máx 500 palavras
	
		O cultivo da soja \lw{Glycine max (L.) Merr.} representa grande importância para o Brasil, por tanto, laboratórios de análise de sementes realizam diversos testes para mensurar as condições de qualidade dos lotes produzidos. Dentre estes testes, destaca-se o teste de tetrazólio que classifica e avalia o vigor e viabilidade de plantio, porém, a etapa de avaliação visual do teste pode gerar subjetividade em alguns casos além de ser cansativa e tediosa para os analistas. 
		O uso de visão computacional para auxiliar a identificação de padrões nas sementes tem impulsionado pesquisas com diversas técnicas em processamento de imagem e diferentes classificadores de dados. Nota-se que a maioria das pesquisas utiliza conjuntos distintos de imagens e não os disponibilizam para que comunidade científica, dificultando a reprodução dos resultados e a comparação com outros métodos sobre os mesmos dados.
		A proposta deste trabalho é elaborar uma metodologia de coleta de dados e desenvolvimento de uma base de imagens publicas, que fomentará pesquisas cientificas no âmbito do teste de tetrazólio em sementes de soja. 
		A metodologia exige uma abordagem simples na etapa de coleta de dados permitindo a aquisição da imagem e a classificação da amostra, alterando o mínimo possível o tempo final da análise quando comparada ao método tradicional. A concordância dos dados coletados devem ser verificados entre os pares, onde uma reavaliação é feita por outros especialistas sobre uma imagem já classificada, então é calculado o coeficiente estatístico de concordância kappa e AC1 dos dados.
		A implementação da metodologia proposta foi realizada com o desenvolvimento de uma ferramenta de coleta e uma plataforma de disponibilização dos dados. Para ferramenta de coleta de dados é utilizado um \ew{software} para dispositivo móvel, como \ew{smartfones} e \ew{tablets} e um sistema(web) para publicização em um banco de dados de imagens, oferecendo uma base verificada e segmentada de dados e imagens de sementes de soja que passaram pelo teste de tetrazólio.
		
		%sistemas de auxílio a agricultura (\ew{CAA - Computer-Aided Agriculture}).
		
			 
	\end{resumo}
		