Toda pergunta precisa de uma resposta, toda resposta precisa de uma fundamentação.



O que é o seu projeto?%perguntas

Qual é o objetivo principal do seu projeto? % o que?


O objetivo é criar uma base
Uma base de informações

Uma base de informações catalográficas
Uma base de informações e imagens

de sementes de soja
Aplicadas/Expostas

ao teste de tetrazólio
ou testadas/analisadas pelo teste de tetrazólio


\emph{Mas a ideia é a Criação de uma base de informações De sementes de soja passadas pelo teste de tetrazólio para apoio a pesquisa}


-> Esse é o seu objetivo: fazer esta base %respostas

Qual é a justificativa para isso? %perguntas
Por que fazer este trabalho?

%lembrei do canvas


A justificativa é muito simples! %respostas


Não existe nenhuma base hoje conhecida sobre sementes e informações das sementes que passaram pelo teste de tetrazólio principalmente contendo imagens % por que

Não tem isso hoje
tudo bem, não tem, mas ...

por que precisa ter? %para que


Por que vários trabalhos são feitos de pesquisa em cima do tetrazólio % apoio a pesquisa

% para resolver um problema
E hoje, vários trabalhos na área de computação estão sendo feito principalmente na área de processamento de imagens
porém quando o pesquisador vai desenvolver algum trabalho
neste sentido ele desenvolve sua própria base de dados
Ele captura sua própria imagens faz a catalogação das suas imagens e faz o estudo


% os problemas
Aí eu consigo ver pelo  menos dois problemas. Pelo menos dois

O primeiro deles: % apelo a um pilar da pesquisa
Se o pesquisador não publica torna publico
Se ele não torna publico o que ele utilizou
Fica muito difícil a reprodução do trabalho
que é muito importante na área de pesquisa
para a pesquisa ser reproduzida


% apelo ao custo da pesquisa
outro problema menos grave % menospresa um problema mesmo grande, para elevar ainda mais o outro
mas também muito importante
é o tempo
não vou usar só o tempo, o custo também
tempo e custo
custo em dinheiro, custo em tempo
custo em esforço
Recursos %apelo aos prazos apertados e baixos financiamentos
%criar a identificação do leitor pesquisador com o problema


para aquisição das imagens
ou se você não tem uma imagem
Não tem uma base
e precisa ter, você tem esse custo a mais

são estes dois motivos: dificuldade de reprodução e custo %Promessa de resolver dois problemas com uma unica soluçao

%aplicação da ideia a um caso específico
essa base precisa ser criada para apoiar futuras pesquisas na área
de tecnologias em sementes principalmente sementes de soja
especificamente usando o teste de tetrazólio
%convenver que a area de aplicação é interessante
%gosto particular
%área importante ao Brasil. Citar o capitulo da dissertação sobre o soja

como fazer isso? %mais perguntas 

primeiro vamos falar de quem já fez e como fez % mais respostas

você vai ter que procurar e citar %embasamento para responder
Vários trabalhos - {\LARGE Isso é importante, vá atrás disso} -
vários trabalhos que fizeram análise de imagem em sementes
de tetrazólio
reconhecimento de padrões
se quiser deixar mais amplo
não fique só no soja
quer ficar mais amplo: não fique só no tetrazólio
talvez seja bom para contextualizar a quantidade de pesquisa feita
com processamento de imagens nesta área para esse publico, para essa área de pesquisa
%ampliou a area de aplicação
%ajuda a justificar a ideia, e expor as possíbilidades onde o projeto pode chegar

% mas já coloca os pé no chão e limita o escopo, porém deixando a
%sensação que trabalhos diferentes podem usar a mesma estratégia
%desperte a atenção exibindo uma oportunidade fácil de trabalhs futuros
tem que fechar isso aí, deixar claro que é só soja e tetrazólio, para não ficar muito amplo
então você tem que procurar esses projetos e esses trabalhos
para amostrar quais bases eles usaram
se a gente conseguir traçar/encontrar
uma base comum que eles esteja usando, a gente referencia

%trabalhos anteriores parecidos e um pouco de informações técnicas
então, projetos que não usam tetrazólio, usam a base X para essa pesquisa
É importante para gente saber a característica desta base
que informações ela tem sobre a semente
e quantas imagem, e formatos, as dimensões da imagem outro ponto
fechando, de tetrazólio como foi feito

Não achando as bases:
Os artigos descrevem suas bases particular
quantas imagens acharam
Quantas de cada classes

Agora a gente procura a maior base, o maior trabalho, sei lá usou mil imagens
Fez o processamento de imagem
procurando aí, sei lá, 4, 5 classes
vamos por 4, 250 imagens por dano
é um numero que dependendo do trabalho, é ok
ou se for 100 por classes, teremos 4000 imagens

cita e fala que é o maior trabalho
informações, por exemplo, qual é o dano mais aparente

Onde está o dano
está segmentado ou não

qual é a espécie, qual é a safra, qual é a região em que foi plantada?
qual é o período em que foi colhida, qual é o peso de mil gramas?
qual é o lote e o resultado final?
para que essas informações? Na minha base tem que ter.

Não vai apoiar pesquisa só para processamento de imagens
apesar de ser o foco
Se eu quiser fazer uma análise estatística
Mostra o tamanho da base e fala
Bom se uma base de 4000 com essas informações
Com uma técnica mais recente de processamento de imagens
necessitaria de muito mais imagens
Isso são para processamentos convencionais, todos esses projetos usaram isso
Outros projetos que demandam muito mais imagens
necessitam de em média de 10.000 imagens por classe
E a gente tem que citar trabalhos que usem essa quantidade de imagens
então já vou
Quando eu falar de importância de base, vamos falar sobre o MNist
que é uma base muito importante
falar o quanto ela foi fundamental e foi usada disseminadamente em vários projetos
O NonMnist dá para falar também
Talvez quando falar de NonMnist, dê para puxar o DeepLearn
que usa muitas imagens
então uma técnica nova
que demanda muita imagem
sua base vai ser
Não existe uma base ainda
deste tamanho, para está aplicação
e você vai fazer, esse é o seu objetivo
Não fazer só uma base, mas fazer uma base grande, muito boa,
completa. Que apoie a pesquisa .
Justamente por que você quer desenvolver um projeto
que classifique a sua base usando deepleaning
Esse era o seu primeiro objetivo
agora você constrói a base, depois você constrói o classificador
fechou
Depois você pede para outro você fazer isso mais pra frente
pro enquanto você vai fazer a base
depois que fizer a base você fala com o próximo
Baseado em projetos
vamos estimas que a gente queira 10 mil imagens
10 mil imagens
por classe, umas 40 mil imagens, 4 classes 40 mil imagens
8 mil imagens
8 mil não, 80 mil imagens
Você vai falar para banca que vai fazer uma base de 80 mil imagens
Fala que vai fazer uma base de 160 mil imagens, o dobro
160 mil imagens
agora tem que calcular o tamanho dessa base
tudo bem, é isso que precisa uma base bem grande
Daí vamos pensar que vaí tirar estas imagens
Se eu fiquei 3 dias para tirar 90, quanto tempo vai precisar?
O negócio é que eu já criei uma forma de coletar
Vamos dividir, vamos colocar muita gente coletando as imagens
Muita gente tirando essas fotos
A estratégia é o seguinte
Quando eu coletei as imagens
Usei uma máquina macro com lentes, estudio
Iluminação controlada, sombra controlada
fiz um procedimento normal e não tive muito sucesso
Não por que as imagens não estavam padronizadas
mas por que tinha pouca imagem
Então a padronização é importante
mas a quantidade também é
E a quantidade é muito mais importante que a padronização dependendo da tecnica
Daí eu preciso que você ache esses dois trabalhos, é importantíssimo que você ache isso aí
existem
Dois, muitos muitos trabalhos. Ache dois e mande
vários trabalhos usando DeepLearning
que você tira uma foto de uma flor
procure no seu quintal uma flor
tire uma foto
e mande para rede neural, deep learning
e ela vai disser qual a espécie daquela flor
tem que treinar com muitas, centenas de imagens
milhares de imagens
para identificar vários tipos de flores
essas fotos não foram padronizadas
justamnete para poder identificar
e determinar esses indicadores
determinar o que realmente indica
a espécie
e isso é o que muda
É uma grande diferença do processamento de imagem tradicional
Onde você implementa vários métodos de extração de características
depois uma rede neural artificial poder classificar
passa por uma rede neural classificador
Ou o deeplearning que você passa a imagem e deixa ela extrair
ela procurar as características
e nisso como é você que faz o trabalho de procura
toda parte
de programação
que você teria para extração das características
não é necessária, a maquina vau fazer isso, deixa ela trabalhar
e aí
pessoas tirando fotos
nos laboratórios hoje
E os analistas enquanto fazem a análise vão tirar fotos dos celulares deles para você
já dá pra ver muito problema aí, então tem que começar a controlar um pouco
primeiro
primeiro que não é tão simples assim, só tirar a foto
eu quero garantir
E você quer também
um pouco de padronização
então pensei em uma forma bem legal
a pessoa vai tirar uma foto
vamos imaginar o seguinte
quanto a pessoa vier tirar
embaixo aqui, tem que ter um gabarito
mas você sabe
eu já sei o que é isso e você sabe
a gente já sabe disso a bastante tempo
quando a gente pensou nisso aqui
estava começando
mas eu vou falar tá
tem que ser um quadrado
um retangulo
vamos fazer em acrílico
recortar em acrílico, vai ter que desenhar isso aí
desenhar, recortar
despachar para todos os laboratórios
talvez você faça algo automontáveis
que você manda pelo correio e a pessoa pega as peças de acrílico, destaca e monta
Próxima parte
Essa caixinha, tem que ter uma tampa transparente e dentro dois espelhos em V
e um espelho assim
90 graus
o espelho
a tampinha em cima
quando a gente coloca a semente aqui em cima
com a base dela aqui
metade metade
colocou aqui virada para baixo
a imagem do fundo dela
vai refletir no fundo dela
que vai refletir neste espelho
que vai refletir para cima, deste lado
então colocou a semente aqui
com a parte de baixo dela
vai bater no espelho
que vai estar na diagonal
vai angular pra ca 90
angulou aqui, sobe
se olhar de cima
vai olhar a semente que você colocou
um semente aqui e vc vê as costa dela
e aqui por espelhos vê o interior dela
e é por aqui que você vai tirar a foto
é só vir com o celular e tirar aqui
resolveu o problema do frente e verso
só que esta caixinha
nem tinha falado como problema
cada semente
tem que ser cortada
lembra na apresentação de falar
Olha só como eu vou colocar
Um monte de pessoas tirando fotos de qualquer jeito
é um problema muito grande
como é a foto
cada lote o cara tem 100 sementes para tirar fotos
o primeiro que você desenvolveu foi um método
o método para tirar essas fotos
para depois sistematizar
quando coloca aqui uma foto já tem todos os lados
evita que tire foto errada e ajuda muito na identificação
quando
você tira uma foto e depois tira outra
para saber que aquelas costas
e aquela outra imagem que está guardada
tem que fazer um alinhamento
tem que pegar o contorno, ver a geometria, o poligono e ver se encaixa
nenhuma eu acho que vai estar
vai precisar rotacionar
pra encaixar certinho
mas esse problema vai acabar
tirar uma foto com a frente e as costas certinha
uma torta a outra torta certinha, igual no mesmo angulo
facilita
muito
e se a pessoa
tirar a foto com o celular, meio assim, assim, assim
então neste seu gabarito
nos 4 cantos vamos colocar marcadores
então por exemplo, vai ser pintado de preto todos os cantos
por der meio centímetro por meio centímetro
dois milímetros por dois milímetros
quando a pessoa for tirar uma foto
e tiver angulado
pelo tamanho, que ficou a marcação em relação a outra
você já sabe se foi ou não foi
se imagem está torta ou não
de qualquer lado que você ver, não precisa nem ter giroscópio
se tiver reto pelo processamento de imagem
depois que a pessoa tira a foto, ela tem que mandar para mim
os danos que a semente tem
e me enviar tudo organizado
160 mil fotos
mas olha, vamos riscar dois zeros por que são 100 por amostras
1.600 amostras
num dia a pessoa faça
umas 5 só
só cinco
160 dias, isso uma pessoa
160mil fotos
tira 2 zeros, 1.600 amostras
agora eu quero saber quantas amostras uma pessoa faz por dia, 5. Então divide por 5
1.600 / 5
320 eu acho
320
320 amostras
só? 320 amostras
não. Você precisa de 1.600 amostras
vai precisa de 320 dias
ou em um dia 32o pessoas
em um dia só
Vamos dizer que você não consiga 320 pessoas
mas você consegue em 10 dias com 32 pessoas
em 10 dias voc? tem a base que você precisa
mas agora imagina 32 pessoas mandando toda essa informação
tem que mandar, tem que organizar, tem que receber certinho, pessoal não pode
não pode atrapalhar o processo dela
e parar para tirar uma foto e mandar tudo isso vai diminuir o rendimento da pessoa
Já sei que é possível fazer
Agora qual é a estratégia para que as pessoas queiram fazer isso por você
essa conta rápida foi para provar que dá
agora eu preciso
desenvolver um processo
para que o analista possa usar, esse processo não pode atrapalhar ele
não pode
não pode tomar mais tempo do que já é gasto
a análise não pode demorar mais do que já demora
senão ele não vai pode fazer
Mas se você está aplicando mais tarefas para fazer
e não quer que aumente o tempo, então
você tem que baixar o tempo das outras tarefas que ele tem
a tarefa que ele tem hoje é
no teste de tetrazólio
a sua estratégia consiste em melhorar o processo original
Em baixar o tempo original, para que
o que ele faz hoje, ele consiga fazer mais rápido
ele vai ter tempo de fazer o dele e mais o seu
e vai poder te ajudar, senão não vai fazer
primeiro que para o cara enviar isso
se demandar de tirar a foto, organizar, planilhar, mandar por email, ou para servidor, usar site
esquece por que não vai funcionar, você vai ter que dar treinamento
vai ter que organizar, vai vir uma base totalmente
você vai fazer um aplicativo de celular que vai usar a câmera do celular e transferir do celular
do aplicativo e já bate as fotos e te envia
colhe os dados, cadastra e manda, vai chegar para você
mas como esse aplicativo vai diminuir o tempo
toda solução dentro de uma aplicativo, fica facil de atualizar
você vai trabalhar com um S.O. só
é suficiente e você sabe disso
o processo deles hoje é bem interessante
eles pegam as 50 sementes e cortam ao meio
as 50 sementes, eles colocam numa solução
de sal de tretrazólio
para agir com a engima do soja
se ela ficar vermelha tem sua análise visual feita pelo técnico treinado
eles cortam cada semente ao meio e analisam o conjunto
quando eles fazer isso eles pegam uma ficha de papel
uma caneta
e tem 50 linhas
50 linhas para cada tipo
por exemplo
dano de umidade, dano mecânico, dano de percevejo
coloca isso na sua apresentação, tem lá no livro do França
então eles vão colocar um simbolozinho
tipo um corte quando um tipo, uma linha quando outro
um tracinho, um X
eles fazem toda análise e vão marcando em uma folha de papel
]no final eles contam
quantos tem de cada, anotam o subtotal
fazem uma média de cada um e dão os resultados
Então este aplicativo vai substítuir este papel
então a pessoa tira uma foto
primeiro captura, depois classifica
próximo
segundo, tira foto, dano tal
e põe alia
ele não vai mais indicar isso escrevendo no papel
ele não vai mais perder o tempo escrevendo no papel, vai usar o tempo para escrever no aplicativo
No final, que ele terminou a analise, você tem os dados que precisa
e eles não tem o relatório dele
tem que ter
então você vai ter que gerar o relatório para ele
a mesma coisa que ele faz, seu aplicativo vai precisar fazer
Qual a vantagem?
você pode far para ele a versão igual a que ele tem
então ele tem o que precisa
nessa estratégia, você consegue que as 3
os 3
interessados, se deem bem
tenham vantagem
um ganha ganha
primeiro ganha o analista
que o processo dele fica mais fácil e mais rápido
usando o aplicativo
Não precisa fazer conta
O relatório está lá pronto para ser impresso e assinado
se precisar imprimir
a empresa ganha
por que a empresa que tem laboratório de sementes
terão documentado de forma digital
Não só em PDF do relatório
que eles não tinham isso
vamo começar a entregar um pouco de beneficio
gerar valor
e depois, eles precisam lançar no sistema interno
tem que registrar isso
toda empresa tem
nosso sistema poderá entregar esses dados formatados, tabulados, em xml idependente do formato
vai pegar um xml e vai entregar
então você recebe as fotos que não tinha antes, que servem para uma auditoria
recebe o relatório digital
que pede ser exatamente como o atual ou com foto
e um xml, que ele pode importar para o sistema dele que ele ganha velocidade
Então o cara usando esse aplicativo a empresa ganha tempo
auditoria
armazenamento de dados, velocidade, integridade, muitas coisas e menos trabalho para ele
que não vai precisar calcular o relatório
que mais?
eu ganho, por que eles vão coletar para mim as informações
recebe um pacote organizado
o pulo do gato está aqui
recebi um monte de imagem e um monte de dados
como eu falo uma base validada para pesquisa?
bom se eu tenho uma classificação e tenho uma imagem
eu posso tentar validar
se está informação é coerente ou não
eu posso pegar no mesmo aplicativo emm outra area
de verificação
que pegue imagens que outra pessoa tirou e envie para aquele analista e pergunte, classifique para mim isso aqui
e vou colocar a classificação dele
