%a soja%
%A participação do Brasil na Produção Internacional%
%A importância da soja para o mercado financeiro e na indústria alimentícia%
%A importância da soja para o Brasil%
%Sobre o controle de produção%
%Sobre os laboratórios de sementes e os teste%
%Sobre o tetrazólio%


\section{A Soja}
A soja cultivada atualmente é muito diferente dos seus ancestrais, que eram plantas rasteiras que se desenvolviam na costa leste da Ásia, principalmente ao longo do rio Yangtse, na China.  Sua evolução começou com o aparecimento  de plantas oriundas de cruzamentos naturais entre duas espécies de soja selvagem que foram domesticadas e melhoradas por cientistas da antiga China.

Cultivada e consumida há milhares de anos pelas civilizações orientais, foi somente a partir do século vinte que foi comercialmente cultivada no Ocidente, mais precisamente nos Estados Unidos(EUA), a partir da década de 1920. Até 1940, a área de soja cultivada para forragem era maior que a cultivada para grãos. A partir de 1941, a área cultivada para grãos superou a cultivada para forragem. \cite{Amelio2011}

\subsection{A participação do Brasil na Produção Internacional}
Segundo \citeonline{FAS2018}, entre os anos de 19xx e 2017, o Brasil liderou a produção de soja no mundo ao lado dos EUA. Atualmente os dois países juntos são responsáveis por 66\% da produção mundial, com 114,10 e 116,92 milhões de toneladas métricas (mmt). A previsão da produção de soja para 2018/19 no Brasil, por \cite{FAS2018}, é um recorde de 117,0 milhões de toneladas métricas (mmt), em uma área colhida prevista para de 36,5 milhões de hectares (mha), 4\% em relação ao ano anterior, prevendo a produtividade de 3,21 toneladas por hectare, mantendo acima da média dos últimos cinco anos. 

\citeauthor{FAS2018} relata que o Brasil deverá continuar sendo o principal exportador de soja em 2018/19, impulsionados pela forte demanda global liderada pela China por alimentos proteicos, estimulando o esmagamento para obtenção do farelo o óleo de soja.

%ver \cite{Junior2017}

\subsection{A importância da soja para o mercado financeiro e na indústria alimentícia}
A commoditie de soja é negociada nas principais bolsas mundias e seu volume de negociações atualmente segundo (????) é de xxx bilhões de toneladas e yyy bilhões de dólares.\nota{Se este paragrafo for se manter, vou procurar os dados e fontes corretas} No Brasil e em vários países o valor da soja é usado para indexar contratos de diversas naturezas envolvendo o agronegócio, principalmente na compra de terras, maquinários agrícolas e empréstimos para financiamento da produção.

O grão de soja tem grande valor na economia global devido suas propriedades nutricionais e na produção de óleos
Usado na alimentação humana e animal a soja fornece grande quantidade de carboidratos e vitaminas, a proteína de soja é muito usada na indústria alimentícia e gastronômica como alternativa a proteína da carne e do leite, atendendo o mercado vegano e de restrições alimentares. A maior utilização da fibra alimentar da soja é na farinha de soja usada em raça animal, muito utilizada na produção de proteína animal, como carne bovina, suína e de frango.

O processo de beneficiamento da soja inclui a extração do óleo vegetal de soja, largamente usado na indústria alimentícia, sendo um dos mais acessíveis para o consumidor final. Segundo \cite{FAS2018a} o óleo de soja é o mais produzido e consumido no mundo entre as oleaginosas. 

\subsection{A importância da soja para o Brasil}
Para atingir tais grandezas de produção a cadeia agroindustrial da soja emprega certa de XXX pessoas no Brasil, e cerca de YYY empresas recolhendo aproximadamente XXX Milhões de reais em impostos, que representou zz\% da arrecadação do setor primário em 2017.\nota{Se este paragrafo for se manter, vou procurar os dados e fontes corretas}

O cultivo da soja no Brasil ao longo dos anos proporcionou a colonização de regiões pouco habitadas no interior do país, promovendo urbanização e crescimento econômico mais distribuído no território nacional. Entretanto cria um desafio para os produtores de soja, que encontram diferentes climas, biomas, solo, ciclos de chuvas e outros fatores, em um país de proporções continentais. Atualmente o Brasil possui registradas xxx cultivares diferentes de soja, sendo zz transgênicas cada cultivar tem suas características como variando a duração das etapas de produção, produção de óleo ou massa seca, quantidade e tamanho dos grãos, estrutura das plantas (tamanho da planta, raízes, volume de folhas), resistência a seca ou ao excesso de água, resistência a herbicidas, muitas variedades transgênicas são modificadas geneticamente para serem nocivas a determinados predadores, diminuindo o uso de inseticidas e fungicidas. Muita da tecnologia brasileira em sementes se origina da EMBRAPA, que atualmente possuem patentes de xxx cultivares de soja transgênica e zzz patentes de cruzamentos naturais.

ver \cite{Junior2017} \cite{LIMA2013} \nota{Material que comecei a ler e posso usar para embasar o texto}

\subsection{Sobre o controle de produção}
Devido a grande importância que a soja representa para nação brasileira é necessário que se proteja o processo de cultivo. Para isso o estado exige uma série de registros e controles sobre a produção, colheita, comercialização, transporte e exportação. Em cada etapa do processo produtivo da soja existem mecanismos de fiscalização por órgãos competentes.
A legislação Brasileira regula quais variedades de soja tem o cultivo permitido no Brasil, com o intuito de controlar a entrada de plantas que podem ser portas de entrada para fungos e insetos, que podem desequilibrar o ecossistema onde local. Situações como esta já aconteceram no Brasil em culturas de cacau, café e banana no passado, prejudicando a economia e o meio ambiente por vários anos

Para garantir autonomia e proteção dos custos de produção, os produtores de soja no Brasil podem produzir suas próprias sementes para safras futuras, mas se houver comercialização de sementes várias normas de qualidade devem ser seguidas, como por exemplo o teste de germinação que determina que se um lote de sementes não possuir o potencial de germinação mais que 70\% das sementes o lote é inviável para o plantio e deverá ser comercializado como grão no mercado secundário.

??? indica que uma produção inferior a 70\% de germinação é inviável em relação aos custos de produção, desta forma a utilização de lotes controlados aumentam as chances de sucesso na safra, diminuindo os riscos que as financiadoras de crédito e seguradoras rurais têm ao liberarem empréstimos aos produtores. Em 2017 xxx Milhões de reais foram liberados para produção de xx\%v das lavouras na Brasil.

\subsection{Sobre os laboratórios de sementes e os testes}
Até 2017 o Brasil possuem cerca de xx mil empresas e cooperativas que comercializam sementes de soja, sendo que xx\% possuem seus próprios laboratórios de análise de sementes, que realizam diversos testes de rotina, afim de determinar a qualidade e detectar as possíveis causas de problemas, para que estes sejam corrigidos. 

Além dos testes exigidos por lei, os testes oficiais de pureza e germinação, é comum as empresas fazer os testes de qualidade como (patologia  tetrazólio - Vigor - Envelhecimento Acelerado - Teste de emergência em areia - condutividade elétrica), para detectar patógenos e determinar inclusive se o processo de colheita e armazenagem foram feitos com qualidade ou causaram danos nas sementes e comprometeram o lote. Determinar a causa da inviabilidade de um lote é vital para ajustar o processo antes que mais lotes sejam comprometidos.

Os laboratórios de sementes empregam mais de x mil analistas no Brasil inteiro, e todo ano instituições como a EMBRAPA formam novas turmas de analistas para atenderem as demandas do mercado. O investimento que as empresas fazem com laboratório, analistas e equipamentos aumentam o custo da semente, mas agregam muito valor na qualidade e garantia das safras seguintes.

\subsection{Sobre o tetrazólio}
Uma das análises mais importantes de um laboratório de sementes de soja é o teste de \sigla{TZ}{Tetrazólio}, que a partir de uma amostra do lote de sementes, tem o objetivo de ressaltar os danos causados em cada semente, para que seja determinada em primeiro momento a viabilidade do lote e a vigorosidade que as plantas terão ao serem plantadas, e em segundo momento se houverem danos nas sementes, determinar o grau dos danos e as causas, que geralmente são causados por excesso de umidade no armazenamento, danos mecânicos do processo de colheita, secagem e estocagem; e danos causados por percevejos que se alimentam dos nutrientes contidos nas sementes.

Para realização do teste cada amostra é submetida a uma solução de sal de tetrazólio por até x horas, dependendo da metodologia usada, durante esta etapa a semente absorve a solução ficando maior e destacando em tons de vermelho carmim os danos que cada semente sofreu. Após a preparação inicial, cada semente é cortada no sentido transversal a radícula para análise do interior e exterior da mesma, na sequência é realizada uma minuciosa analise visual de cada metade da semente em busca de padrões característicos dos danos mencionados, os danos de cada semente são anotados em uma ficha de controle e ao final o analista calcula os resultados que classificam o lote.

As limitações do teste de tetrazólio, citadas por FRANÇA-NETO (1998) incluem a exigência de um treinamento especial sobre a estrutura embrionária da semente, experiência e paciência pois a análise é relativamente tediosa. Atualmente o mercado exige cada vez mais profissionais capacitados em realizar o teste.


Dada a importância que o teste de tetrazólio tem no processo de garantia de sucesso no cultivo da soja, e a importância da cadeia produtiva no Brasil e no mundo, é justificável que pesquisas sejam aplicadas com o objetivo de tornar o processo de análise mais rápido e assertivo.

